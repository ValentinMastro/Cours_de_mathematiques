\documentclass[11pt]{exam}
\usepackage{listings}
\usepackage[margin=2cm]{geometry}

%%%%%%%%%%%%%%%%%%%%%%%%%%%%%%%%%%%%%%%%%%% Coloration
\usepackage{xcolor}

%New colors defined below
\definecolor{codegreen}{rgb}{0,0.6,0}
\definecolor{codegray}{rgb}{0.5,0.5,0.5}
\definecolor{codepurple}{rgb}{0.58,0,0.82}
\definecolor{backcolour}{rgb}{0.95,0.95,0.92}

%Code listing style named "mystyle"
\lstdefinestyle{mystyle}{
  backgroundcolor=\color{backcolour}, commentstyle=\color{codegreen},
  keywordstyle=\color{magenta},
  numberstyle=\tiny\color{codegray},
  stringstyle=\color{codepurple},
  basicstyle=\ttfamily\footnotesize,
  breakatwhitespace=false,         
  breaklines=true,                 
  captionpos=b,                    
  keepspaces=true,                 
  numbers=left,                    
  numbersep=5pt,                  
  showspaces=false,                
  showstringspaces=false,
  showtabs=false,                  
  tabsize=2
}

%"mystyle" code listing set
\lstset{style=mystyle}
%%%%%%%%%%%%%%%%%%%%%%%%%%%%%%%%%%%%%%%%%%%%%

\begin{document}
\begin{questions}
\qformat{\textbf{Q\thequestion)\hfill}}

\question Compléter le code ci-dessous pour que la fonction \emph{somme} retourne la somme de tous les arguments $a$, $b$ et $c$.

\begin{lstlisting}[language=Python]
def somme(a, b, c):
    return 
\end{lstlisting}

\question Compléter le code ci-dessous pour que la fonction \emph{signe} qui prend pour argument $x$ retourne $+1$ si $x$ est positif, $-1$ si $x$ est négatif et $0$ sinon.

\begin{lstlisting}[language=Python]
def signe(x):
    if (           ):
        return 
    elif (           ):
        return 
    elif (           ):
        return
\end{lstlisting}

\question Compléter le code ci-dessous pour que la fonction \emph{signe\_produit} qui prend pour arguments $a$ et $b$ retourne le signe du produit $a\times b$ ($+1$ si c'est positif, $-1$ si c'est négatif, $0$ sinon). Vous devez réutiliser la fonction \emph{signe} vue à la question précédente.

\begin{lstlisting}[language=Python]
def signe_produit(a,b):
    return 
\end{lstlisting}

\question Compléter le code ci-dessous pour que la fonction \emph{duree} qui prend en arguments $j$, $h$, $m$ et $s$ et qui renvoie le nombre de secondes qui s'est écoulé durant $j$ jours, $h$ heures, $m$ minutes et $s$ secondes.

\begin{lstlisting}[language=Python]
def duree(j, h, m, s):
    jours_en_secondes = 
    heures_en_secondes = 
    minutes_en_secondes = 
    
    return jours_en_secondes + heures_en_secondes + minutes_en_secondes + s
\end{lstlisting}

\question Écrire une fonction \emph{annee\_bissextile} qui prend en argument une année $n$ et qui renvoie $True$ si une année est bissextile et $False$ sinon.\\
Une année est bissextile si elle est multiple de 4, sauf lorsqu'elle est multiple de 100, mais elle l'est quand même si elle est multiple de 400.\\[1em]
Tester la fonction avec les valeurs 2000, 2100, 2022, 2023.

\question Écrire une fonction \emph{fizzbuzz} qui prend en argument un entier $n$ et qui affiche (avec $print$) : $Fizz$ si $n$ est multiple de 3, $Buzz$ si $n$ est multiple de $5$, $FizzBuzz$ si $n$ est multiple de $3$ et de $5$, l'entier $n$ sinon.\\[1em]
Pour tester votre programme, utilisez le code suivant :
\begin{lstlisting}[language=Python]
for i in range(0,100):
    fizzbuzz(i)
\end{lstlisting}



\end{questions}
\end{document}