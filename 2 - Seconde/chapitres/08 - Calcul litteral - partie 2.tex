\documentclass[../Cours.tex]{subfiles}

\color{bleu}

\begin{document}
\chapitre{Inégalités et inéquations}

\partie{Les inégalités}

\notation{Soient $a$ et $b$ deux nombres réels, on note
\begin{itemize}
    \item $a \textless b$ la proposition $a$ est strictement inférieur à $b$
    \item $a \leq b$ la proposition $a$ est inférieur ou égal à $b$
    \item $a \textmore b$ la proposition $a$ est strictement supérieur à $b$
    \item $a \geq b$ la proposition $a$ est plus petit que $b$
\end{itemize}
}
\souspartie{Notation positionnelle décimale}

\vocabulaire{Un chiffre est un symbole qui permet, en les assemblant, de construire des nombres.\\ Il en existe 10 : \\\centerline{0 ; 1 ; 2 ; 3 ; 4 ; 5 ; 6 ; 7 ; 8 ; 9}}



\partie{Nombres décimaux}

\end{document}