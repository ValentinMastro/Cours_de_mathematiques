\documentclass[../Cours.tex]{subfiles}
\begin{document}

\chapitre{Calcul littéral}


\partie{Expression littérale}

\souspartie{Écriture}
\definition{Une \emph{expression littérale} est un ensemble d'opérations contenant des nombres et des lettres.}
\definition{Dans une expression littérale, une lettre représente un nombre, cette lettre est appelée une \emph{variable}.}

\exemple{$3\times a + 2 \times b$ est une expression littérale contenant 2 variables : $a$ et $b$. Elle est composée de 3 opérations : 2 multiplications et 1 addition.}

\definition{Évaluer une expression signifie remplacer chacune des variables par une valeur numérique donnée.}

\exemple{Évaluer l'expression $3 \times a + 8$ en $a = 2$ signifie remplacer $a$ par 2, ce qui donne :  $3 \times 2 + 8 = 14$.}

\convention{%
Le symbole de multiplication est facultatif entre :
\begin{itemize}
    \item deux variables : $a\times b = ab$
    \item un nombre et une variable : $5 \times x = 5x$
    \item avec une parenthèses : $2 \times (x+3) = 2(x+3)$
\end{itemize}
}

\notation{\vspace{-1.5em}\begin{itemize}
        \item $a \times a = a^2$ (se lit << $a$ au carré >>)
        \item $a \times a \times a = a^2 \times a = a^3$ (se lit << $a$ au cube >>)
    \end{itemize}
}

\begin{listedexemples}
    \item $5^2 = 5 \times 5 = 25$
    \item $2^3 = 2 \times 2 \times 2 = 8$
    \item $h^3 = h \times h \times h$
    \item $(a+b)^2 = (a+b)(a+b)$
\end{listedexemples}

\souspartie{Réduction}

\definition{Réduire une expression, c'est regrouper les termes de même nature/classe/famille.}

\exemple{$2x+3x = 5x$}

\partie{Développement et factorisation}

\souspartie{Somme et produit}

\definition{Une expression littérale est une \emph{somme} lorsque la dernière opération est une addition ou une soustraction.}

\exemple{$3a+b$ / $7(2+x)+y$}

\definition{De même, si la dernière opération est une multiplication, on dit que c'est un \emph{produit}.}

\exemple{$3x$ $7(x+y)$ $(a+b)(2a+3b)$}

\souspartie{Distributivité simple}

\definition{Développer, c'est transformer un produit en une somme.}
\definition{Factoriser, c'est transformer une somme en un produit}

\propriete{\textsc{DISTRIBUTIVITÉ}\\ 
\begin{center}
    \begin{tikzpicture}
        \node[anchor=west] at (0,0) {$k(a+b) = ka + kb$};
        \node[anchor=west] at (0,-1.5) {$ka + kb = k(a+b)$};
        \draw[vert,-latex] (0.25,0.3) arc (175:5:0.25);
        \draw[vert,-latex] (0.25,0.3) arc (175:5:0.65 and 0.45);
        \draw[vert] (0.27,-1.5) circle (0.2 and 0.4);
        \draw[vert] (1.45,-1.5) circle (0.2 and 0.4);
    \end{tikzpicture}
\end{center}
}

\end{document}