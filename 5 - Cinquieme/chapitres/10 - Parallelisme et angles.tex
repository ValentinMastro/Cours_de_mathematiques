\documentclass[../Cours.tex]{subfiles}
\begin{document}

\chapitre{Angles et parallélisme}

\partie{Les paires d'angles}

\illustration{Deux droites $(d)$ et $(d')$ sont coupées par une troisième droite $(\Delta)$.
\begin{center}
    \begin{tikzpicture}[scale=1.8]
        \tikzmath{ \r=0.3; }
        \draw[name path=d1] (0,0) node[left]{$(d)$} -- (7,0.8);
        \draw[name path=d2] (-0.5,-1) node[left]{$(d')$} -- (8,-1.2);
        \draw[name path=d3] (4,1.2) node[above]{$(\Delta)$} -- (1.5,-2);
        \path[name intersections={of=d1 and d3, by=A}];
        \path[name intersections={of=d2 and d3, by=B}];
        %\foreach \i/\j/\c in {-2/52/rouge, 52/178/vert, 178/232/jaune, 232/358/blue} { \fill[\c] (B) -- ($(B)+({\r*cos(\i)},{\r*sin(\i)})$) arc(\i:\j:\r) -- cycle;}
        %\foreach \i/\j/\c in {-174/-128/noir, -128/6/orange, 6/52/pink, 52/186/red} { \fill[\c] (A) -- ($(A)+({\r*cos(\i)},{\r*sin(\i)})$) arc(\i:\j:\r) -- cycle; }
        \node [above left] at (A) {\pgfmathparse{Hex(10101+1)}\LARGE \libertineGlyph{uni\pgfmathresult}};
        \node [above right] at (A) {\pgfmathparse{Hex(10101+2)}\LARGE \libertineGlyph{uni\pgfmathresult}};
        \node [below left] at (A) {\pgfmathparse{Hex(10101+3)}\LARGE \libertineGlyph{uni\pgfmathresult}};
        \node [below right] at (A) {\pgfmathparse{Hex(10101+4)}\LARGE \libertineGlyph{uni\pgfmathresult}};
        \node [above left] at (B) {\pgfmathparse{Hex(10101+5)}\LARGE \libertineGlyph{uni\pgfmathresult}};
        \node [above right] at (B) {\pgfmathparse{Hex(10101+6)}\LARGE \libertineGlyph{uni\pgfmathresult}};
        \node [below left] at (B) {\pgfmathparse{Hex(10101+7)}\LARGE \libertineGlyph{uni\pgfmathresult}};
        \node [below right] at (B) {\pgfmathparse{Hex(10101+8)}\LARGE \libertineGlyph{uni\pgfmathresult}};
    \end{tikzpicture}
\end{center}
}

\souspartie{Opposés par le sommet}

\vocabulaire{
Deux angles sont \emph{opposés par le sommet} s'ils ont le même sommet, et que leurs côtés sont dans le prolongement l'un de l'autre.
}

\exemple{Dans le schéma précédent, sont opposés par le sommet :
\begin{itemize}
    \item les angles 1 et 4
    \item les angles 2 et 3
    \item les angles 5 et 8
    \item les angles 6 et 7
\end{itemize}
}

\propriete{Deux angles opposés par le sommet ont la même mesure.}

\souspartie{Correspondants}

\vocabulaire{
On considère deux droites coupées par une sécante. Deux angles sont \emph{correspondants} si :
\begin{itemize}
    \item ils sont du même côté de la sécante
    \item ils ont des sommets différents
    \item l'un est à l'intérieur de la zone formée par les deux droites, l'autre est à l'extérieur
\end{itemize}
}

\remarque{Deux angles correspondants << regardent dans la même direction >>.}

\exemple{Dans le schéma précédent, sont correspondants :
\begin{itemize}
    \item les angles 1 et 5
    \item les angles 3 et 7
    \item les angles 6 et 2
    \item les angles 4 et 8
\end{itemize}
}

\souspartie{Alternes-internes}

\vocabulaire{On considère deux droites coupées par une sécante. Deux angles sont \emph{alternes-internes} si :
\begin{itemize}
    \item ils ne sont pas du même côté de la sécante
    \item ils ont des sommets différents
    \item ils sont tous les deux à l'intérieur de la zone délimitée par les deux droites
\end{itemize}
}

\exemple{Dans le schéma précédent, sont alternes-internes :
\begin{itemize}
    \item les angles 3 et 6
    \item les angles 5 et 4
\end{itemize}
}

\partie{Lien avec le parallélisme}

\propriete{Les angles correspondants sont égaux $\Longleftrightarrow$ les droites sont parallèles.}
\propriete{Les angles alternes-internes sont égaux $\Longleftrightarrow$ les droites sont parallèles.}

\end{document}