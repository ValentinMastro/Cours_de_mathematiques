\documentclass[../Cours.tex]{subfiles}
\begin{document}

\chapitre{Divisibilité et nombres premiers}
\partie{Multiples et diviseurs}
\souspartie{Division euclidienne}

\definition{Soient deux nombres entiers $n$ et $p$. Il existe deux nombres $q$ et $r$ tels que : $$n = p \times q + r ~~~~~~(\mbox{avec}~~ 0 \infeg r < q)$$}

\exemple{%
\begin{center}
    \begin{tikzpicture}[scale=0.7]
        \node[anchor=south west] at (0,0) {
            \begin{tabularx}{0.17\textwidth}{CCC|CC}
                  & 6 & 5 & 4 &   \\
                - & 4 &   & 1 & 6 \\
                  & 2 & 5 &   &   \\
                - & 2 & 4 &   &   \\
                  &   & 1 &   &   \\
            \end{tabularx}
        };
        \draw (0.5,3) -- (2,3);
        \draw (0.5,1.1) -- (2.89,1.1);
        \draw (2.89,3.8) -- (4,3.8);
        \draw[noir,-Latex] (-2,4.2) -- (1,4.2); 
        \draw[noir,-Latex] (7,4.2) -- (4,4.2);
        \draw[noir,-Latex] (-2,0.7) -- (2,0.7);
        \draw[noir,-Latex] (7,3.4) -- (4.8,3.4);
        \node[anchor=east] at (-2,0.7) {\color{noir} Reste};
        \node[anchor=east] at (-2,4.2) {\color{noir} Dividende};
        \node[anchor=west] at (7,4.2) {\color{noir} Diviseur};
        \node[anchor=west] at (7,3.4) {\color{noir} Quotient};
        \node at (2.8,-2) {\Large{\color{rouge} \fbox{$65 = 4 \times 16 + 1$}}};
    \end{tikzpicture}
\end{center}
}

\souspartie{Critère de divisibilité}

\definition{On dit qu'un nombre $n$ est divisible par $q$ si le reste de la division euclidienne $n\div q$ vaut 0.}

\begin{listedexemples}
    \item $42$ est divisible par \textcolor{rouge}{6} car $42 = \textcolor{rouge}{6} \times 7$.
    \item $35$ est divisible par \textcolor{rouge}{5} car $35 = \textcolor{rouge}{5} \times 7$.
\end{listedexemples}

\paragraphe{noir}{Critères de divisibilité}{
    \begin{tabularx}{\textwidth}{l|C|l}
        & règle & exemples \\\hline
        divisible par 2 & \makecell{le nombre est pair \\ le chiffre des unités est 0, 2, 4, 6 ou 8} & \makecell{\textcolor{vert}{102 est pair}\\ \textcolor{rouge}{113 est impair}} \\\hline
        divisible par 5 & le chiffre des unités est 0 ou 5 & \makecell{\textcolor{vert}{\num{1000000005}} \\ \textcolor{rouge}{22} } \\\hline
        divisible par 10 & le chiffre des unités est 0 & \makecell{ \textcolor{vert}{\num{4000000000}} \\ \textcolor{rouge}{13} } \\\hline
        divisible par 3 & la somme des chiffres est dans la table de 3 & \makecell{\textcolor{vert}{213 $\longrightarrow 2+1+3 = 6$}\\\textcolor{rouge}{$17 \longrightarrow 1+7 = 8$}} \\\hline
        divisible par 9 & la somme des chiffres est dans la table de 9 & \makecell{\textcolor{vert}{342 $\longrightarrow 3+4+2 = 9$}\\\textcolor{rouge}{$14 \longrightarrow 1+4 = 5$}} \\\hline
    \end{tabularx}
}

\partie{Nombres premiers}
\souspartie{Concept}

\definition{Un nombre premier est un nombre entier ($\neq 1$) pour lequel les seuls diviseurs sont $1$ et lui-même.}

\paragraphe{noir}{Liste des nombres premiers $\infeg 30$}{2 ; 3 ; 5 ; 7 ; 11 ; 13 ; 17 ; 19 ; 23 ; 29}

\begin{listedexemples}
    \item 7 est un nombre premier car ses seuls diviseurs sont 1 et 7.
    \item 6 n'est pas un nombre premier car 2 et 3 sont des diviseurs de 6. \\ En effet, $6 = 2 \times 3$.
\end{listedexemples}

\souspartie{Décomposition en facteurs premiers}
\theoreme{fondamental de l'arithmétique}{Tout nombre entier peut se décomposer de manière unique (à l'ordre des facteurs près) en un produit de nombres premiers.}

\begin{listedexemples}
    \item $35 = 5 \times 7$
    \item $216 = 2 \times 2 \times 2 \times 3 \times 3 \times 3 $
\end{listedexemples}


\end{document}