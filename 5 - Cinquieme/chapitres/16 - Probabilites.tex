\documentclass[../Cours.tex]{subfiles}
\begin{document}

\chapitre{Probabilités}

\partie{Concept}

\definition{Une expérience est aléatoire si je ne peux pas déterminer à l'avance l'issue de l'évènement.}

\exemple{ << Lancer un dé à 6 faces >> est un évènement aléatoire, car je ne peux pas savoir à l'avance sur quoi va tomber le dé avant de le lancer.\\
Les issues de cette expérience sont $\Omega = {1;2;3;4;5;6}$.}

\definition{La probabilité d'un évènement est un nombre entre 0 et 1 qui permet de quantifier la possibilité que l'évènement se réalise.}

\partie{Approche qualitative}

\begin{center}
    \begin{tikzpicture}
        \node[rouge,text width=3.5cm] at (-6,0) {évènement impossible};
        \node[rouge,text width=3.5cm] at (-2,0) {évènement peu probable};
        \node[rouge,text width=3.5cm] at (2,0) {évènement probable};
        \node[rouge,text width=3.5cm] at (6,0) {évènement certain};
        \draw[dashed] (-4,1) -- (-4,-3);
        \draw[dashed] (0,1) -- (0,-3);
        \draw[dashed] (4,1) -- (4,-3);
        \draw[-latex] (-8,-1) -- (8,-1);
        \node[noir, text width=3.5cm] at (-6,-2) {lancer un dé et obtenir un 0};
        \node[noir, text width=3.5cm] at (-2,-1.5) {gagner au loto};
        \node[noir, text width=3.5cm] at (-2,-2.5) {avoir un 1 en lançant un dé};
        \node[noir, text width=3.5cm] at (2,-1.5) {perdre au loto};
        \node[noir, text width=3.5cm] at (2,-3) {avoir entre 2 et 6 en lançant un dé};
        \node[noir, text width=3.5cm] at (6,-2.5) {lancer un dé et avoir un nombre entre 1 et 6};
    \end{tikzpicture}
\end{center}    

\partie{Approche quantitative}

\propriete{La probabilité d'une issue est égale à $\dfrac{1}{\card({\Omega})}$,\\ $\card({\Omega})$ est le nombre d'issues au total.}

\begin{listedexemples}
    \item Pile ou face : $\Omega = \left\{ \mbox{"pile"} ; \mbox{"face"} \right\}$, $\card{(\Omega)} = 2$\\
    $p\left( \mbox{pile} \right) = \dfrac{1}{2}$
    \item Lancer de dé : $\Omega = \left\{ 1;2;3;4;5;6 \right\}, \card{(\Omega)} = 6 $\\
    $p\left( \mbox{avoir un 3} \right) = \dfrac{1}{6}$
    \item On choisit une lettre au hasard dans l'alphabet :\\
    $\Omega = \left\{ A; B; C; D; E; F; G; H; I; J; K; L; M; N; O; P; Q; R; S; T; U; V; W; X; Y; Z \right\}$\\
    $\card{(\Omega)} = 26$\\
    $p\left( \mbox{avoir un H} \right) = \dfrac{1}{26}$
\end{listedexemples}

\definition{Un évènement est un ensemble d'issues.}

\exemple{Pour l'expérience aléatoire << lancer un dé >> :\\
A = << obtenir un nombre pair >> $= \left\{ 2;4;6 \right\}$\\
B = << obtenir un nombre premier >> $= \left\{ 2;3;5 \right\} $}

\propriete{La probabilité d'un évènement A est égale à $\dfrac{\card{(A)}}{\card{(\Omega)}}$}

\exemple{Dans l'expérience aléatoire de lancer un dé, $\Omega = \left\{ 1;2;3;4;5;6 \right\}$.\\ 
L'évènement A = << obtenir un nombre pair >> $= \left\{ 2;4;6 \right\} \longrightarrow \card{(A)} = 3$\\
$p(A) = \dfrac{3}{6}$}

\clearpage
\EXERCICES
\begin{questions}
\exercice Dans un jeu de cartes de 32 cartes classiques, Victor choisit au hasard une carte dans le paquet.
\question Donner la liste de toutes les issues possibles ?
\question Quel est le nombre d'issues possibles ?
\question Quelle est la probabilité de tirer un valet de carreau ?
\question Quelle est la probabilité de tirer une carte trèfle ?

\exercice Une urne opaque contient trois boules rouges et deux boules bleues, indiscernables au toucher. Laure tire au hasard une boule dans l'urne.
\question Quelle est la probabilité de tirer une boule rouge ?
\question Quelle est la probabilité de tirer une boule bleue ?
\question Donner les deux résultats précédents sous forme de pourcentage.

\exercice Une roue est divisée en trois secteurs de couleurs différentes. Jérôme fait tourner la roue.

\def\centerarc[#1](#2)(#3:#4:#5)% Syntax: [draw options] (center) (initial angle:final angle:radius)
    { \draw[#1] ($(#2)+({#5*cos(#3)},{#5*sin(#3)})$) arc (#3:#4:#5); }


\question Pourquoi peut-on dire que les trois couleurs ont autant de chances d'être obtenues ?
\question Quelle est la probabilité d'obtenir une des trois couleurs ? Donner le résultat sous forme de fraction, de nombre décimal et de pourcentage.


\begin{center}
    \begin{tikzpicture}[scale=1.4]
        \draw[fill=noir] (0,1) -- (0.1,1.2) -- (-0.1,1.2) -- cycle;
        \draw[rouge,fill=rouge] (0,0) -- (1,0) arc(0:120:1) -- cycle;
        \draw[bleu, fill=bleu] (0,0) -- (1,0) arc(0:-120:1) -- cycle;
        \draw[vert, fill=vert] (0,0) -- ({cos(120)},{sin(120)}) arc(120:240:1) -- cycle;
    \end{tikzpicture}
\end{center}

\exercice Sur les 6 faces d'un dé standard, on inscrit les lettres du mot TOUPIE.
\question Quelles sont les issues de l'expérience ?
\question Donner la probabilité de chacune des lettres possibles sous forme de fraction.

\exercice Deux urnes contiennent des boules indiscernables au toucher. on choisit une des deux urnes et on tire une boule au hasard. On gagne si la boule obtenue est rouge.

\begin{center}
\begin{tikzpicture}
    \draw (0,0) rectangle (3,2);
    \draw[rouge,fill=rouge] (0.5,1.4) circle (0.2) node[right of=0.5,anchor=west] {35 boules rouges};
    \draw[jaune,fill=jaune] (0.5,0.6) circle (0.2) node[right of=0.5,anchor=west] {65 boules jaunes};
    \draw (5,0) rectangle +(3,2);
    \draw[rouge,fill=rouge] (5.5,1.4) circle (0.2);
    \draw[jaune,fill=jaune] (5.5,0.6) circle (0.2);
\end{tikzpicture}
\end{center}

\end{questions}

\end{document}