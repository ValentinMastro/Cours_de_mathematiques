\documentclass[../Cours.tex]{subfiles}
\begin{document}

\chapitre{Parallélogramme}



\partie{Nomenclature des parallélogrammes}

\begin{center}
    \begin{tikzpicture}
        \draw (-2,11) rectangle (2,13);
        \node at (0,12) {quadrilatères};
        \draw (-2,7) rectangle (2,9);
        \node at (0,8) {parallélogrammes};
        \draw (-8,3) rectangle (-4,5);
        \node at (-6,4) {losange};
        \draw (4,3) rectangle (8,5);
        \node at (6,4) {rectangle};
        \draw (-2,-1) rectangle (2,1);
        \node at (0,0) {carré};
        \draw[vert,-Latex] (0.25,11) -- (0.25,9);
        \draw[rouge,-Latex] (-0.25,11) -- (-0.25,9);
        \draw[vert,-Latex] (-2,8) -- (-6,5);
        \draw[vert,-Latex] (2,8) -- (6,5);
        \draw[vert,-Latex] (-6,3) -- (-2,0);
        \draw[vert,-Latex] (6,3) -- (2,0);
        \draw[rouge,-Latex] (-2,8.25) -- (-6.5,5);
        \draw[rouge,-Latex] (2,8.25) -- (6.5,5);
        \draw[rouge,-Latex] (-6.5,3) -- (-2,-0.25);
        \draw[rouge,-Latex] (6.5,3) -- (2,-0.25);
        \node[anchor=west,vert] at (0.25,10) {\small \makecell{les côtés opposés\\ sont parallèles}};
        \node[anchor=east,rouge] at (-0.25,10) {\small \makecell{les diagonales se coupent\\ en leur milieu}};
        \node[vert,rotate=35] at (-3.6,6) {\small \makecell{les 4 côtés sont\\ égaux}};
        \node[rouge,rotate=35] at (-4.8,7) {\small \makecell{les diagonales sont\\ perpendiculaires}};
        \node[vert,rotate=-35] at (3.3,6.2) {\small \makecell{les 4 angles sont \\égaux (=\ang{90})}};
        \node[rouge,rotate=-35] at (4,7.5) {\small \makecell{les diagonales sont\\ de même longueur}};
        \node[vert,rotate=35] at (3.5,2) {\small \makecell{les 4 côtés sont\\ égaux}};
        \node[rouge,rotate=35] at (4.8,1) {\small \makecell{les diagonales sont\\ perpendiculaires}};
        \node[vert,rotate=-35] at (-3,1.8) {\small \makecell{les 4 angles sont \\égaux (=\ang{90})}};
        \node[rouge,rotate=-35] at (-5,1) {\small \makecell{les diagonales sont\\ de même longueur}};
    \end{tikzpicture}
\end{center}

\souspartie{Losange}
\definition{Un losange est un parallélogramme qui a 4 côtés égaux.}
\illustration{}
\propriete{Dans un losange, les diagonales sont perpendiculaires.}

\souspartie{Rectangle}
\definition{Un rectangle est un parallélogrammes dont tous les angles sont égaux à \ang{90}.}
\illustration{}
\propriete{Dans un rectangle, les diagones sont de même longueur.}

\souspartie{Carré}
\definition{Un parallélogramme qui a les mêmes propriétés que le losange et le rectangle est un carré.}
\illustration{}


\clearpage
\EXERCICES
\begin{questions}
    \exercicetitre{Constructions}
        \question Construire en vraie grandeur le rectangle $IJKL$ tel que $IJ=\qty{5}{\centi\metre}$ et $JL=\qty{8}{\centi\metre}$.
        \question Construire en vraie grandeur le losange $EFGH$ tel que $EG=\qty{3}{\centi\metre}$ et $\widehat{EHG}=\ang{30}$.
    \exercicetitre{Programme de construction}
        \question Écrire un programme de construction pour $GHFE$ sachant que $OG=\qty{2}{\centi\metre}$ et $OH=\qty{3}{\centi\metre}$.
        \begin{center}
        \begin{tikzpicture}
            \coordinate (u) at (3,0.3);
            \coordinate (v) at (0.5,2);
            \draw (0,0) node[left]{$F$} -- ++(u) node[midway]{||} node[right]{$E$} -- ++(v) node[midway,rotate=90]{||} node[right]{$H$} -- ++($-1*(u)$) node[midway]{||} node[left]{$G$} -- cycle node[midway,rotate=90]{||};
            \draw[dashed] (0,0) -- ++($(u)+(v)$) node[midway,above]{$O$};
            \draw[dashed] (0,0) ++(v) -- ++($(u)-(v)$);
        \end{tikzpicture}
        \end{center}
        \question Écrire un programme de construction pour $KLMN$ sachant que $LM=\qty{3.5}{\centi\metre}$ et $\widehat{KVN}=\ang{30}$.
        \begin{center}
        \begin{tikzpicture}
            \coordinate (u) at (6,0);
            \coordinate (v) at (0,1.5);
            \draw (0,0) node[left]{$N$} -- ++(u) node[right]{$M$} -- ++(v)  node[right]{$L$} -- ++($-1*(u)$)  node[left]{$K$} -- cycle;
            \draw[fill=black] (0,0) rectangle ++(0.2,0.2);
            \draw[fill=black] (u) rectangle ++(-0.2,0.2);
            \draw[fill=black] (v) rectangle ++(0.2,-0.2);
            \draw[fill=black] ($(u)+(v)$) rectangle ++(-0.2,-0.2);
            \draw[dashed] (0,0) -- ++($(u)+(v)$) node[midway,above]{$V$};
            \draw[dashed] (0,0) ++(v) -- ++($(u)-(v)$);
        \end{tikzpicture}
        \end{center}
\end{questions}

\begin{questions}
    \exercicetitre{Constructions}
        \question Construire en vraie grandeur le rectangle $IJKL$ tel que $IJ=\qty{5}{\centi\metre}$ et $JL=\qty{8}{\centi\metre}$.
        \question Construire en vraie grandeur le losange $EFGH$ tel que $EG=\qty{3}{\centi\metre}$ et $\widehat{EHG}=\ang{30}$.
    \exercicetitre{Programme de construction}
        \question Écrire un programme de construction pour $GHFE$ sachant que $OG=\qty{2}{\centi\metre}$ et $OH=\qty{3}{\centi\metre}$.
        \begin{center}
        \begin{tikzpicture}
            \coordinate (u) at (3,0.3);
            \coordinate (v) at (0.5,2);
            \draw (0,0) node[left]{$F$} -- ++(u) node[midway]{||} node[right]{$E$} -- ++(v) node[midway,rotate=90]{||} node[right]{$H$} -- ++($-1*(u)$) node[midway]{||} node[left]{$G$} -- cycle node[midway,rotate=90]{||};
            \draw[dashed] (0,0) -- ++($(u)+(v)$) node[midway,above]{$O$};
            \draw[dashed] (0,0) ++(v) -- ++($(u)-(v)$);
        \end{tikzpicture}
        \end{center}
        \question Écrire un programme de construction pour $KLMN$ sachant que $LM=\qty{3.5}{\centi\metre}$ et $\widehat{KVN}=\ang{30}$.
        \begin{center}
        \begin{tikzpicture}
            \coordinate (u) at (6,0);
            \coordinate (v) at (0,1.5);
            \draw (0,0) node[left]{$N$} -- ++(u) node[right]{$M$} -- ++(v)  node[right]{$L$} -- ++($-1*(u)$)  node[left]{$K$} -- cycle;
            \draw[fill=black] (0,0) rectangle ++(0.2,0.2);
            \draw[fill=black] (u) rectangle ++(-0.2,0.2);
            \draw[fill=black] (v) rectangle ++(0.2,-0.2);
            \draw[fill=black] ($(u)+(v)$) rectangle ++(-0.2,-0.2);
            \draw[dashed] (0,0) -- ++($(u)+(v)$) node[midway,above]{$V$};
            \draw[dashed] (0,0) ++(v) -- ++($(u)-(v)$);
        \end{tikzpicture}
        \end{center}
\end{questions}

\end{document}