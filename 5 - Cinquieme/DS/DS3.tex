\documentclass[../Cours.tex]{subfiles}

\usepackage{xstring}
\usepackage{xfp}

\begin{document}
\clearpage
\thispagestyle{empty}

\setcounter{DS}{2}

\color{black}
\nomPrenom
\titreDS

\begin{questions}
    \EXERCICETITRE{6}{Symétries}
    \question Construire un rectangle $ABCD$ de largeur \qty{3}{\centi\metre} et de longueur \qty{2}{\centi\metre}.
    \question Construire la droite $(d)$ parallèle à la diagonale $[AC]$ passant par $D$.
    \question Construire le symétrique $A'B'C'D'$ de $ABCD$ par rapport à la droite $(d)$
    \question \textbf{Sur une autre figure,} tracer un triangle rectangle $DEF$ rectangle en $E$. (vous pouvez choisir les longueurs du triangle)
    \question Construire le triangle $D'E'F'$ symétrique de $DEF$ par rapport au point $E$ (symétrie centrale)
    
    \EXERCICETITRE{4}{Boite de conserve}
    Une boite de conserve cylindrique a pour rayon $r=\qty{4.95}{\centi\metre}$ et pour hauteur $h=\qty{11.8}{\centi\metre}$.
    \question Calculer le volume de cette boite (rappel : $V_{\mbox{cylindre}} = \pi \times r \times r \times h$)
    \question Convertir le résultat précédent en \unit{\litre} puis en \unit{\centi\litre}.

    \EXERCICETITRE{4}{Téléviseurs}
    La plupart des téléviseurs dans le commerce sont au format 16:9, ce qui signifie que la longueur et la largeur du téléviseur sont dans le ratio 16:9.

    \question Si un de ces téléviseurs à une longueur de \qty{32}{\centi\metre}, quelle sera sa largeur ?
    \question Si un de ces téléviseurs à une largeur de \qty{45}{\centi\metre}, quelle sera sa longueur ?
    \question Si un téléviseur a pour périmètre \qty{250}{\centi\metre}, quelles seront sa longueur et sa largeur ?

    \EXERCICETITRE{4}{Eau salée}
    Pour faire cuire des pâtes, on les introduit dans l'eau bouillante quelques minutes. Il est recommandé d'utiliser \qty{10}{\gram} de sel pour \qty{1}{\kilo\gram} d'eau. On rappelle

    \question Si on utilise \qty{2.7}{\kilo\gram} d'eau, combien de sel doit-on utiliser ?
    \question Si on a utilisé \qty{2}{\gram} de sel, quel quantité d'eau faut-t-il faire bouillir ?

    \EXERCICETITRE{2}{Multiples et diviseurs}

    \question Donner \textbf{tous} les diviseurs de 110.
        
    
\end{questions}
\end{document}