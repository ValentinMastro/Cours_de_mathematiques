\documentclass[addpoints,12pt]{exam}

\usepackage[locale=FR, group-digits=all, group-separator=\ , group-minimum-digits=4, per-mode = symbol]{siunitx}
\usepackage[margin=1.5cm]{geometry}
\usepackage{mdframed}
\usepackage{tikz}
\usepackage{frenchmath}
\DeclareSIUnit{\litre}{\ell}
\usepackage{amsmath}
\usepackage{eurosym}
\DeclareSIUnit{\EURO}{\text{\euro}}

\newcommand{\titre}[1]{%
    % #1 : numéro du DS
    \begin{center}%
        \Huge{DS n°#1}%    
    \end{center}%
}

\begin{document}

\titre{1}

\qformat{\textbf{\underline{Exercice \thequestion : }} \hfill [\thepoints]}
\bonusqformat{\textbf{\underline{Exercice \thequestion : }} \hfill [\thepoints]}
\begin{questions}
    \question[3]
    Pierre a sauvegardé 150 fichiers de musique (*.mp3), 250 images (*.jpeg) et 300 fichiers vidéos (*.mp4) sur un disque dur externe.

    \begin{parts}
        \part Quelle est la proportion de fichiers vidéos sur son disque dur ?
        \part Si chaque fichier de musique pèse \qty{2}{Mo}, chaque image pèse \qty{1}{Mo} et chaque vidéo \qty{100}{Mo}, combien de Mo pèse tous les fichiers ensemble ?
    \end{parts}

    \question[4]
    \begin{parts}
        \part Pour chacun des triangles, dire s'ils sont constructibles.
        \begin{subparts}
            \subpart $ABC$ : $AB = 2$, $BC = 6$, $AC = 9$
            \subpart $DEF$ : $DF = 4$, $DE = 8$, $EF = 5$
        \end{subparts}
        \part Dans chacun des triangles, trouver l'angle manquant.
        \begin{subparts}
            \subpart $IJK$ : $\widehat{IJK} = \ang{50}$, $\widehat{IKJ} = \ang{70}$
            \subpart $NOP$ : $\widehat{NOP} = \ang{22,5}$, $\widehat{NPO} = \ang{98,1}$
        \end{subparts}
    \end{parts}

    \question[4]
    \begin{parts}
        \part Tracer un triangle $XYZ$ tel que $XY = \qty{3}{\cm}$, $YZ = \qty{4}{\cm}$, $XZ = \qty{5}{\cm}$
        \part Tracer les trois médiatrices du triangle $XYZ$
    \end{parts}

    \question[3]
    Un groupe d'amis décident de se partager un gâteau au chocolat.\\
    Alice a mangé $\dfrac{1}{7}$ du gâteau, Bob a mangé $\dfrac{10}{21}$ du gâteau et Céline a mangé $\dfrac{16}{42}$ du gâteau. Que restera-t-il pour Dominique ?

    \question[2]
        Un pavé droit a les dimensions suivantes : $L = \qty{7}{\centi\metre}$, $\ell = \qty{3}{\centi\metre}$, $h = \qty{2}{\centi\metre}$. \\
        Calculer son volume.

    \question[4]
    Soit $IJK$ un triangle isocèle en $I$ avec $IJ = \qty{3}{\cm}$ et $JK = \ang{50}$.
    \begin{parts}
        \part Tracer le triangle $IJK$
        \part Démontrer que la hauteur issue de $I$ et la médiatrice de $[JK]$ sont superposées. \\ (autrement dit : que ce sont les mêmes droites)
    \end{parts}

    \bonusquestion[1] Donner le 278ème chiffre après la virgule de $\dfrac{16,3}{999}$.
\end{questions}

\end{document}