\documentclass[10pt]{article}
\usepackage[a5paper,margin=1cm]{geometry}
\usepackage{tikz}
\usetikzlibrary{calc,math}

\newcounter{VMQCMNumeroQuestion}
\newcounter{VMQCMTotalPoints}
\setcounter{VMQCMNumeroQuestion}{20}
\setcounter{VMQCMTotalPoints}{20}

\pagestyle{empty}

\begin{document}
    standalone
    \vfill
    \begin{tikzpicture}
        % Données élèves
        % NIVEAU (6ème, 5ème, etc.)
        \node[anchor=east] at (0,0) {\footnotesize{\textit{Niveau}}};
        \tikzmath{
            coordinate \p;
            for \y in {1,2,...,4}{
                \p1 = (0,-0.3*\y+0.1);
                \p2 = (0.2,-0.3*\y-0.1);
                \p3 = (\p1)-(0,0.1);
                {\node[anchor=east] at (\p3) {\tiny{\directlua{tex.print(math.floor(7.5-\y))}ème}};};
                {\draw (\p1) rectangle (\p2);};
            };
        }
        % CLASSE (A, B, C, etc.)
        \node[anchor=east] at (1.3,0) {\footnotesize{\textit{Classe}}};
        \tikzmath{
            int \lettre;
            coordinate \p;
            for \y in {1,2,...,6}{
                \p1 = (1.3,-0.3*\y+0.1);
                \p2 = (1.5,-0.3*\y-0.1);
                \p3 = (\p1)-(0,0.1);
                \lettre = 65-1+\y;
                {\node[anchor=east] at (\p3) {\tiny{\char\lettre}};};
                {\draw (\p1) rectangle (\p2);};
            };
        }
        % NUMÉRO de l'élève
        \node[anchor=east] at (2.9,0) {\footnotesize{\textit{Numéro}}};
        \tikzmath{
            coordinate \p;
            coordinate \q;
            int \y;
            for \y in {0,1,...,9}{
                \p1 = (2.3,{-0.3*(\y+1)+0.1});
                \p2 = (\p1)+(0.2,-0.2);
                \p3 = (\p1)+(0,-0.1);
                \q1 = (\p1)+(0.8,0);
                \q2 = (\q1)+(0.2,-0.2);
                \q3 = (\q1)+(0,-0.1);
                {\draw (\p1) rectangle (\p2);};
                {\draw (\q1) rectangle (\q2);};
                {\node[anchor=east] at (\p3) {\tiny{\y}};};
                {\node[anchor=east] at (\q3) {\tiny{\y}};};
            };
        }
        % Grille de réponse
        \tikzmath{
            int \lettre;
            coordinate \O;
            int \n;
            \O1 = (5.1,-0.6);
            \O2 = ({5.1+0.3*\theVMQCMNumeroQuestion},-1.801);
            {\draw (\O1) grid[shift={(0,0)},step=0.3] (\O2);};
            coordinate \p;
            coordinate \l;
            for \n in {1,2,...,\theVMQCMNumeroQuestion}{
                \p1 = (5+0.3*\n-0.075, {-0.65-1.5*mod(\n+1,2)});
                {\node[anchor=south] at (\p1) {\tiny{\n}};};
                if \n<5 then {
                    \l1 = (5.1,{-0.3*(\n+2)+0.15});
                    \lettre = 65-1+\n;
                    {\node[anchor=east] at (\l1) {\tiny{\char\lettre}};};
                };
            };
        }
        % Encadré pour la note
        \draw (-1.1,-2.7) rectangle (1.1,-3.2);
        \node at (0,-3) {Note : ~~~~/\theVMQCMTotalPoints};
    \end{tikzpicture}
\end{document}
