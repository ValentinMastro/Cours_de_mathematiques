\documentclass[../Cours.tex]{subfiles}

\begin{document}

\chapitre{Nomenclature et notation géométriques}

\partie{Droite, demi-droite, segment}

\vocabulaire{Une \emph{droite} est un objet géométrique formé de points alignés. Elle se prolonge à l'infini des deux côtés.}
\notation{Le nom d'une droite s'écrit entre parenthèses.}

\exemple{
    \begin{figure*}[h]
        \centering
        \color{noir}
        \begin{tikzpicture}
            \coordinate (u) at (0,0);
            \coordinate (v) at (4,1);
            \draw (u) -- (v);
            \coordinate (A) at ($(u)!0.2!(v)$);
            \coordinate (B) at ($(u)!0.8!(v)$);
            \node at (A) {|};
            \node at (B) {|};
            \node[below right] at (A) {$A$};
            \node[below right] at (B) {$B$};
        \end{tikzpicture}
        \caption{La droite $(AB)$}
    \end{figure*}
}

\vocabulaire{Une \emph{demi-droite} est un morceau d'une droite délimité par un point : l'origine.}
\notation{Une demi-droite d'origine $A$ et de direction $x$ se note $[Ax)$. Une demi-droite d'origine $B$ et passant par $O$ se note $[BO)$.}

\exemple{
    \begin{figure*}[h]
        \centering
        \color{noir}
        \begin{tikzpicture}
            \coordinate (A) at (-1,2);
            \coordinate (x) at (3,1);
            \draw (A) -- (x);
            \node at (A) {|};
            \node[above right] at (A) {$A$};
            \node[above] at (x) {$x$};
        \end{tikzpicture}\hspace{3cm}
        \begin{tikzpicture}
            \coordinate (B) at (1,0);
            \coordinate (O) at (5,1);
            \draw (B) -- (O) -- ($(B)!1.2!(O)$);
            \node at (B) {|};
            \node at (O) {|};
            \node[above right] at (B) {$B$};
            \node[above left] at (O) {$O$};
        \end{tikzpicture}
        \caption{La demi-droite $[Ax)$ et la demi-droite $[BO)$}
    \end{figure*}
}

\clearpage

\vocabulaire{Un segment est une portion de droite délimités par deux points : ses extrémités.}
\notation{On note les deux extrémités du segment entre crochets.}

\exemple{
    \begin{figure*}[h]
        \centering
        \color{noir}
        \begin{tikzpicture}
            \coordinate (U) at (0,0);
            \coordinate (V) at (4,1);
            \draw (U) -- (V);
            \node at (U) {|};
            \node at (V) {|};
            \node[below right] at (U) {$U$};
            \node[below right] at (V) {$V$};
        \end{tikzpicture}
        \caption{Le segment d'extrémités $U$ et $V$ : $[UV]$ }
    \end{figure*}
}

\partie{Polygone}

\definition{Un polygone est une figure fermée composée de plusieurs segments, appelés côtés.}

\nomenclature{\vspace{-1em}%
\begin{itemize}
    \item 3 côtés $\longrightarrow$ triangle
    \item 4 côtés $\longrightarrow$ quadrilatère
    \item 5 côtés $\longrightarrow$ pentagone
    \item 6 côtés $\longrightarrow$ hexagone
    \item 8 côtés $\longrightarrow$ octogone
    \item 10 côtés $\longrightarrow$ décagone
\end{itemize}
}

\definition{Les sommets sont les extrémités communes à 2 côtés d'un polygone.}

\notation{Les sommets sont des points, donc ils s'écrivent en majuscule.}

\clearpage
\partie{Points et appartenance}

\notation{Un point se note avec une lettre majuscule.}

\begin{listedexemples}
    \item Le point $A$
    \item Les points $C$ et $D$
    \item Le point $\Omega$
\end{listedexemples}

\notation{Lorsqu'un point appartient à un objet géométrique (une droite, un segment, etc.), on utilise le symbole $\in$.}

\exemple{
\begin{figure}[h!]
    \centering
    \begin{tikzpicture}
        \coordinate (A) at (0,0);
        \coordinate (B) at (2.5,1);
        \coordinate (C) at (2,-1);
        \coordinate (D) at (4,0.8);
        \coordinate (E) at ($(A)!0.6!(B)$);
        \coordinate (F) at ($(C)!1.3!(D)$);
        \coordinate (H) at (-0.2,-1);
        \draw[noir] (A) -- (B);
        \draw[noir] (C) -- ($(C)!1.6!(D)$);
        \draw ($(A)+(-0.1,-0.1)$) -- ($(A)+(0.1,0.1)$);
        \draw ($(A)+(0.1,-0.1)$) -- ($(A)+(-0.1,0.1)$);
        \draw ($(B)+(-0.1,-0.1)$) -- ($(B)+(0.1,0.1)$);
        \draw ($(B)+(0.1,-0.1)$) -- ($(B)+(-0.1,0.1)$);
        \draw ($(C)+(-0.1,-0.1)$) -- ($(C)+(0.1,0.1)$);
        \draw ($(C)+(0.1,-0.1)$) -- ($(C)+(-0.1,0.1)$);
        \draw ($(D)+(-0.1,-0.1)$) -- ($(D)+(0.1,0.1)$);
        \draw ($(D)+(0.1,-0.1)$) -- ($(D)+(-0.1,0.1)$);
        \node[above left] at (A) {$A$};
        \node[above left] at (B) {$B$};
        \node[above left] at (C) {$C$};
        \node[above left] at (D) {$D$};
        \draw[rouge] ($(E)+(-0.1,-0.1)$) -- ($(E)+(0.1,0.1)$);
        \draw[rouge] ($(E)+(0.1,-0.1)$) -- ($(E)+(-0.1,0.1)$);
        \draw[rouge] ($(F)+(-0.1,-0.1)$) -- ($(F)+(0.1,0.1)$);
        \draw[rouge] ($(F)+(0.1,-0.1)$) -- ($(F)+(-0.1,0.1)$);
        \draw[rouge] ($(H)+(-0.1,-0.1)$) -- ($(H)+(0.1,0.1)$);
        \draw[rouge] ($(H)+(0.1,-0.1)$) -- ($(H)+(-0.1,0.1)$);
        \node[above left,rouge] at (E) {$E$};
        \node[above left,rouge] at (F) {$F$};
        \node[above left,rouge] at (H) {$H$};
        \node[anchor=west] at (7,1) {$E \in [AB]$};
        \node[anchor=west] at (7,0.3) {$H \notin [AB]$};
        \node[anchor=west] at (7,-0.4) {$F \notin [CD]$};
        \node[anchor=west] at (7,-1.1) {$F \in [CD)$};
    \end{tikzpicture}
\end{figure}
}



\end{document}