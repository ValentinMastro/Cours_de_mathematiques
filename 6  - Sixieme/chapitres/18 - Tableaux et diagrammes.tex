\documentclass[../Cours.tex]{subfiles}

\begin{document}
\chapitre{Diagrammes}

\partie{Diagrammes en bâtons}

\definition{Un diagramme en bâtons ou en barres permet de représenter les données d'un tableau par des barres proportionnelles aux données.}

\exemple{}
\begin{center}%
Nombre d'habitants dans 3 pays en 2020\\
\begin{tabular}{|c|c|}\hline
    \textsc{Pays} & \textsc{Nombre d'habitants} (en millions) \\\hline
    France & 67 \\\hline
    Pologne & 38 \\\hline
    Corée du sud & 51 \\\hline
\end{tabular}%
\end{center}%

\begin{flushright}
    \tiny{Sources : Instituts nationaux officiels de statistiques}
\end{flushright}


\begin{center}
    \begin{tikzpicture}
        \draw[->] (-1,0) -- (11,0);
        \draw[->] (0,-1) -- (0,8);
        \foreach \y in {10,20,...,70} {
            \draw (-0.2,{0.1*\y}) -- (0.2,{0.1*\y});
            \node[left] at (-0.2,{0.1*\y}) {\y};
        }
        \node[below] at (2,0) {France};
        \node[below] at (5,0) {Pologne};
        \node[below] at (8,0) {Corée du Sud};
        \draw[fill=noir] (1.8,0) rectangle (2.2,6.7);
        \draw[fill=noir] (4.8,0) rectangle (5.2,3.8);
        \draw[fill=noir] (7.8,0) rectangle (8.2,5.1);
        \draw[dashed] (0,6.7) -- (1.8,6.7);
        \draw[dashed] (0,3.8) -- (4.8,3.8);
        \draw[dashed] (0,5.1) -- (7.8,5.1);
        \node[thin, above left] at (0,5.1) {\small{51}};
        \node[thin, left] at (0,6.7) {\small{67}};
        \node[thin, below left] at (0,3.8) {\small{38}};
    \end{tikzpicture}
\end{center}

\clearpage
\partie{Diagrammes circulaires}

\definition{Un diagramme circulaire permet de représenter les données d'un tableau par des secteurs de disque d'angles proportionnels aux données du tableau.}

\exemple{}

\begin{center}
    Estimation du nombre d'habitants par continent en 2021\\
    \begin{tabular}{|c|c||c|}\hline
        \textsc{Continent} & \makecell{\textsc{Nombre d'habitants}\\ (en millions)} & \textsc{Angle} (en °) \\\hline
        Asie & 4 679 & 213 \\\hline
        Afrique & 1 373 & 62 \\\hline
        Europe & 747 & 34 \\\hline
        Amérique latine et Caraïbes & 659 & 30 \\\hline
        États-Unis et Canada & 371 & 16 \\\hline
        Océanie & 43 & 1 \\\hline 
        Antartique & 0 & 0 \\\hline
        \textsc{TOTAL} & 7872 & 360 \\\hline 
    \end{tabular}
\end{center}

\begin{flushright}
    \tiny{Source : https://www.un.org/development/desa/publications/world-population-prospects-the-2017-revision.html}
\end{flushright}

\vspace{1cm}

\newcommand{\macroLegende}[3]{ % position, nom, couleur
    \draw[fill=#3] (4,#1) rectangle (4,{#1+0.3});
    \node[above right] (4,#1) {#2};
}

\begin{center}
    \begin{tikzpicture}
        \draw (0,0) circle (3);
        \foreach \y in {-1.5,-1,...,1.5} {
            \draw (4.5,\y) -- (8,\y);
        }
        
        \macroLegende{-1.5}{}{}
        
        \draw (4,-1) rectangle (4.3,-0.7);
        \draw (4,-0.5) rectangle (4.3,-0.2);
        \draw (4,0) rectangle (4.3,0.3);
        \draw (4,0.5) rectangle (4.3,0.8);
        \draw (4,1) rectangle (4.3,1.3);
        \draw (4,1.5) rectangle (4.3,1.8);
        
        \draw[fill=noir] (0,0) circle (0.03);
        \draw[very thin] (0,0) -- (3,0);
    \end{tikzpicture}
\end{center}

\clearpage
\begin{questions}
    \exercice\\ Ces graphiques indiquent l'ensoleillement par moi à Lille au cours des années 2016 et 2017.
    
    \begin{tikzpicture}
        % Axe des abscisses
        \coordinate (A) at (-0.75,0);
        \coordinate (decal) at (1.5,0);
        \draw (0,0) -- (18,0);
        
        \newcommand{\dataA}[1]{\ifcase#1\or62,1\or80,2\or121,5\or163,6\or197,8\or124,9\or211,4\or243,2\or179,8\or115\or74,5\or80,9\fi}
        \newcommand{\dataB}[1]{\ifcase#1\or81,5\or50,1\or153,9\or201,2\or216,6\or254,7\or197,2\or183,5\or137,5\or101,7\or89,5\or26,3\fi}

        \foreach \x in {1,2,...,12} {
            \node[below] at ($(A)+\x*(decal)$) {\tiny{\DTMfrenchmonthname{\x}}};
            \draw[fill=rouge] ($(A)-(0.3,0)+\x*(decal)$) rectangle ($(A)+(0.3,0)+\x*(decal)+(0,{\dataA{\x}/10})$);
        }
    \end{tikzpicture}
    
    \clearpage
    \exercice 
\end{questions}




\end{document}