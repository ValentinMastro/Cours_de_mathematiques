\documentclass[../Cours.tex]{subfiles}

\begin{document}
\chapitre{Gestion de données\\ Lire et représenter des données}

\partie{Avec un tableau}

Pour représenter des données, on peut utiliser un tableau.\\
Un tableau comporte des \emph{lignes} (horizontales) et des \emph{colonnes} (verticales).\\

\exemple{Le tableau ci-dessous représente le développement d'un oeuf d'autruche en fonction de la température.
\begin{center}
    \begin{tabular}{|c|c|c|c|c|}\hline
    Température (en degrés Celcius) & \num{36.5} & \num{36} & \num{35.5} & \num{35} \\\hline   
    Durée de développement (en jours) & 40 & 42 & 44 & 47 \\\hline
    \end{tabular}
\end{center}
\begin{enumerate}
    \item Combien de lignes et de colonnes contient ce tableau ?
    \item Qu'indique la première ligne ?
    \item Qu'indique la deuxième ligne ?
    \item S'il fait 36°C, combien de temps l’œuf mettra-t-il à se développer ?
    \item Si l'œuf a mis 44 jours à se développer, à quelle température était-il placé ?
\end{enumerate}
\begin{enumerate}
    \item Ce tableau contient deux lignes et cinq colonnes.
    \item La première ligne indique la température de développement.
    \item La deuxième ligne indique la durée de développement de l’œuf.
    \item S’il fait 36°C, l’œuf mettra 42 jours à se développer.
    \item Si l’œuf a mis 44 jours à se développer, il était placé à 35,5°C.
\end{enumerate}
}

\partie{Avec un graphique ou un diagramme}

Pour représenter l'évolution d'une grandeur par rapport à une autre, on utilise un graphique cartésien.

\exemple{Le graphique ci-dessous représente l'évolution de la 
hauteur d'une tige de blé en fonction de son âge.
\begin{enumerate}
    \item Que lit-on sur l'axe vertical ?
    \item Que lit-on sur l'axe horizontal ?
    \item Au bout de 6 jours, combien mesure la tige ?
    \item Si la tige mesure 10 mm, quel âge a le blé ?
\end{enumerate}}



\end{document}