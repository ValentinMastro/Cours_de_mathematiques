\documentclass[../Cours.tex]{subfiles}

\begin{document}

\setcounter{chapitre}{-1}
\chapitre{Les nombres : vers la notation positionnelle}

\partie{Chiffres et nombres}

\definition{Il existe 10 \emph{chiffres} dans un système décimal : 0;1;2;3;4;5;6;7;8;9. En les combinant, on peut former des \emph{nombres}.}

\exemple{Avec les chiffres 1 et 2, je peux former les nombres 1 ; 2 ; 12 ; 21.}

\propriete{La valeur représentée par un chiffre dépend de sa position par rapport à la virgule.}

{
\small
\newcommand*\rot{\rotatebox{90}}
\begin{center}
\begin{tabularx}{\textwidth}{|*{13}{C|}}\hline
    \multicolumn{3}{|c}{Millions} & \multicolumn{3}{|c}{Milliers} & \multicolumn{3}{|c|}{Unités} & , & \multicolumn{3}{|c|}{Millièmes} \\\hline
    \rot{centaine de millions} & \rot{dizaine de millions} & \rot{millions} & \rot{centaine de milliers} & \rot{dizaine de milliers} & \rot{milliers} & \rot{centaines} & \rot{dizaines} & \rot{unités} & & \rot{dixièmes} & \rot{centièmes} & \rot{millièmes} \\\hline
\end{tabularx}
\end{center}
}

\begin{listedexemples}
    \item $45,219 = (4 \times 10) + (5 \times 1) + (2 \times 0,1) + (1 \times 0,01) + (9 \times 0,001)$
    \item $\num{17845,6} = 1 \times \num{10 000} + 7 \times \num{1 000} + 8 \times 100 + 4 \times 10 + 5 \times 1 + 6 \times 0,1$
    \item $\num{2,456} =$ \color{rouge} $ 2 \times 1 + \frac{4}{10} + \frac{5}{100} + \frac{6}{1000} =$ \color{vert} $ 2 \times 1 + 4 \times 0,1 + 5 \times 0,01 + 6 \times 0,001$ \color{bleu}
\end{listedexemples}

\partie{Comparaison de nombres}

\definition{Comparer deux nombres consiste à dire lequel est le plus grand ou le plus petit. On utilise les symboles $<$, $>$, $\infeg$, $\supeg$ et $=$.}

\begin{listedexemples}
    \item $45,5 < 45,68$
    \item $0,02 \supeg 0,01$
\end{listedexemples}


\end{document}