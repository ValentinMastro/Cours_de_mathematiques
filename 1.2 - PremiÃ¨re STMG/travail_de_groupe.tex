\documentclass[11pt]{memoir}
\usepackage{importationsprojet}

\thispagestyle{empty}

\begin{document}
\begin{questions}

\exercice 
\question Donner les dix premiers termes des suites suivantes :
\subpart la suite $(u_n)$ arithmétique de premier terme $u_0 = 3$ et de raison 7
\subpart la suite $(v_n)$ géométrique de premier terme $v_0 = 0,5$ et de raison 2
\subpart la suite $(w_n)$ géométrique de premier terme $w_0 = 3$ et de raison 3
\subpart la suite $(t_n)$ définie explicitement par $t_n = n^2+1$

\question Donner tous les termes des suites $(u_n)$, $(v_n)$, $(w_n)$ et $(t_n)$...
\subpart valant 3
\subpart valant 10

\exercice
\question Soit la suite $(u_n)$ définie telle que $u_n = 3n+2$. Montrer que la suite $(u_n)$ est croissante.
\question Soit la suite $(w_n)$ définie telle que $w_n = -5n+2$. Montrer que la suite $(v_n)$ est décroissante.

\exercice \\
Sur un livret A, on place \qty{1000}{\text{\euro}} le 1er janvier 2023.\\
Tous les ans au 1er janvier, l'argent sur ce compte augmente de 1\% grâce aux intérêts.\\

On décide de représenter l'argent sur le compte par une suite $(a_n)$, on note $a_0$ l'argent sur le compte le 1er janvier 2023, $a_1$ l'argent sur le compte le 1er janvier 2024, etc.\\

\question Combien vaut $a_0$ ?
\question Calculer $a_1$, $a_2$, $a_3$
\question De quel type de suite s'agit-il ?
\question À quelle année, y aura-t-il \qty{1100}{\text{\euro}} sur le compte ?

\exercice \\
Dans une autre banque, le banquier vous propose de déposer sur un autre compte les \qty{1000}{\text{\euro}} et vous dit : <<~Tous les ans, au 1er janvier, je rajouterai \qty{11}{\text{\euro}} d'intérêts sur votre compte.~>>\\

On décide de représenter l'argent sur le compte par une suite $(b_n)$, on note $b_0$ l'argent sur le compte le 1er janvier 2023, $b_1$ l'argent sur le compte le 1er janvier 2024, etc.\\

\question Combien vaut $b_0$ ?
\question Calculer $b_1$, $b_2$, $b_3$
\question De quel type de suite s'agit-il ?
\question À quelle année, y aura-t-il \qty{1100}{\text{\euro}} sur le compte ?

\exercice \\
On définit la suite $(u_n)$ telle que :
\begin{itemize}
    \item $u_0 = 10$
    \item si $u_n$ est pair, alors $u_{n+1} = \dfrac{u_n}{2}$
    \item si $u_n$ est impair, alors $u_{n+1} = 3\times u_n + 1$
\end{itemize}

\question Calculer les termes de $(u_n)$ jusqu'à ce qu'un des termes vaille 1.
\question Représenter graphiquement la suite $(u_n)$.


\end{questions}
\end{document}