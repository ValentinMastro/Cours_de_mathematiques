\documentclass[../Cours.tex]{subfiles}

\begin{document}
\chapitre{Divisions}

\partie{Division euclidienne}

\propriete{Soient deux nombres entiers que l'on appelle dividende et diviseur. Alors il existe deux nombres entiers uniques appelés quotient et reste tels que :
$$\mbox{dividende} = \mbox{diviseur} \times \mbox{quotient} + \mbox{reste}$$
\centerline{avec la condition : $0 \infeg \mbox{reste} < \mbox{quotient}$}}

\exemple{
\begin{center}
    \begin{tikzpicture}[scale=0.7]
        \node[anchor=south west] at (0,0) {
            \begin{tabularx}{0.17\textwidth}{CCC|CC}
                  & 6 & 5 & 4 &   \\
                - & 4 &   & 1 & 6 \\
                  & 2 & 5 &   &   \\
                - & 2 & 4 &   &   \\
                  &   & 1 &   &   \\
            \end{tabularx}
        };
        \draw (0.5,3) -- (2,3);
        \draw (0.5,1.1) -- (2.89,1.1);
        \draw (2.89,3.8) -- (4,3.8);
        \draw[noir,-Latex] (-2,4.2) -- (1,4.2); 
        \draw[noir,-Latex] (7,4.2) -- (4,4.2);
        \draw[noir,-Latex] (-2,0.7) -- (2,0.7);
        \draw[noir,-Latex] (7,3.4) -- (4.8,3.4);
        \node[anchor=east] at (-2,0.7) {\color{noir} Reste};
        \node[anchor=east] at (-2,4.2) {\color{noir} Dividende};
        \node[anchor=west] at (7,4.2) {\color{noir} Diviseur};
        \node[anchor=west] at (7,3.4) {\color{noir} Quotient};
        \node at (2.8,-2) {\Large{\color{rouge} \fbox{$65 = 4 \times 16 + 1$}}};
    \end{tikzpicture}
\end{center}
}

\partie{Division décimale}

\methode{On commence comme la division euclidienne.\\Dès qu'il n'y a plus de chiffres du dividende à faire descendre, on ajoute une virgule et des zéros.\\On s'arrête lorsque le reste vaut 0.}

\exemples{
\begin{center}
\opdiv[displayintermediary=all,voperation=top,dividendbridge]{65}{4}
\end{center}
}




\end{document}