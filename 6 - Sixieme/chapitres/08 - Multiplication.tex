\documentclass[../Cours.tex]{subfiles}

\begin{document}
\chapitre{Multiplication}

\partie{Multiplier et diviser par 1,10,100,etc.}

\regle{Pour multiplier un nombre décimal par 1,10,100,etc., on décale la virgule vers la droite en fonction du nombre de zéros.}

\begin{listedexemples}
    \item $\num{2,12345} \times 100 = \num{212,345}$
    \item $\num{32,45678} \times 10000 = \num{324567,8}$
    \item $\num{7,2} \times 100 = 720$
\end{listedexemples}

\regle{Pour diviser un nombre par 1, 10, 100, etc., on décale la virgule vers la gauche en fonction du nombre de zéros.}

\begin{listedexemples}
    \item $\num{3247,2347} \div 100 = \num{32,472347}$
    \item $\num{1249,7} \div 1000 = \num{1,24497}$
    \item $\num{12} \div 10000 = \num{0,0012}$
\end{listedexemples}

\begin{center}
\begin{tabularx}{\textwidth}{l|*{7}{|C}|}\hline
    préfixe & kilo & hecto & déca & $\vide$ & déci & centi & milli\\\hline
    distance (en mètre) & $\unit{\kilo\metre}$ & $\unit{\hecto\metre}$ & $\unit{\deca\metre}$ & $\unit{\metre}$ & $\unit{\deci\metre}$ & $\unit{\centi\metre}$ & $\unit{\milli\metre}$ \\
    masse (en kilogramme) & $\unit{\kilo\gram}$ & $\unit{\hecto\gram}$ & $\unit{\deca\gram}$ & $\unit{\gram}$ & $\unit{\deci\gram}$ & $\unit{\centi\gram}$ & $\unit{\milli\gram}$ \\
    durée (en seconde) & $\unit{\kilo\second}$ & $\unit{\hecto\second}$ & $\unit{\deca\second}$ & $\unit{\second}$ & $\unit{\deci\second}$ & $\unit{\centi\second}$ & $\unit{\milli\second}$ \\
    température (en Kelvin) & $\unit{\kilo\kelvin}$ & $\unit{\hecto\kelvin}$ & $\unit{\deca\kelvin}$ & $\unit{\kelvin}$ & $\unit{\deci\kelvin}$ & $\unit{\centi\kelvin}$ & $\unit{\milli\kelvin}$ \\
    \makecell[l]{intensité du courant \\ électrique (en Ampère)} & $\unit{\kilo\ampere}$ & $\unit{\hecto\ampere}$ & $\unit{\deca\ampere}$ & $\unit{\ampere}$ & $\unit{\deci\ampere}$ & $\unit{\centi\ampere}$ & $\unit{\milli\ampere}$ \\
    intensité lumineuse (en candela)& $\unit{\kilo\candela}$ & $\unit{\hecto\candela}$ & $\unit{\deca\candela}$ & $\unit{\candela}$ & $\unit{\deci\candela}$ & $\unit{\centi\candela}$ & $\unit{\milli\candela}$ \\
    quantité de matière (en mole)& $\unit{\kilo\mole}$ & $\unit{\hecto\mole}$ & $\unit{\deca\mole}$ & $\unit{\mole}$ & $\unit{\deci\mole}$ & $\unit{\centi\mole}$ & $\unit{\milli\mole}$ \\\hline
\end{tabularx}
\end{center}


\underline{Les préfixes du Système International (SI) :}
\begin{multicols}{2}
\begin{itemize}
    \item kilo : un millier $\Rightarrow \times 1000$ 
    \item hecto : une centaine $\Rightarrow \times 100$
    \item déca : une dizaine $\Rightarrow \times 10$
    \item milli : un millième $\Rightarrow \div 1000$
    \item centi : un centième $\Rightarrow \div 100$
    \item déci : un dixième $\Rightarrow \div 10$
\end{itemize}
\end{multicols}

\partie{Multiplication décimale}

\regle{Pour multiplier deux nombres décimaux, on effectue la multiplication en ignorant la virgule. À la fin, on place la virgule pour qu'il y ait autant de chiffres après la virgule qu'à l'origine.}

\exemple{Calculer $2,72 \times 3,8$
\begin{center}
    \begin{tikzpicture}
        \node[anchor = south west] at (0,0) {%
        \begin{tabularx}{0.15\linewidth}{CCCCCC}
              &   & 2 & , & 7 & 2 \\
            $\times$  &   &   & 3 & , & 8 \\\hline
              &   & 2 & 1 & 7 & 6 \\
            $+$  &   & 8 & 1 & 6 & 0 \\\hline
            1 & 0 & , & 3 & 3 & 6 \\  
        \end{tabularx}};
        \draw[vert,Latex-] (3.5,3) -- (5,3);
        \draw[vert,Latex-] (3.5,2.4) -- (5,2.4);
        \draw[vert,Latex-] (3.5,0.5) -- (5,0.5);
        \node[vert,anchor=west,font={\small}] at (5,3) {2 chiffres après la virgule};
        \node[vert,anchor=west,font={\small}] at (5,2.4) {1 chiffre~ après la virgule};
        \node[vert,anchor=west,font={\small}] at (5,0.5) {3 chiffres après la virgule};
    \end{tikzpicture}
\end{center}}

\clearpage
\begin{questions}
    \exercice Multiplier ou diviser par 1, 10, 100, etc.
        \begin{multicols}{3}
        \begin{luacode}
            for i = 1,15
            do
                nombre = int(math.random(10,999) / math.pow(10, math.random(-2,3)))
                facteur = int(math.pow(10, math.random(1,4)))
                
                operateurs = {"\\times", "\\div"}
                op = objet_aleatoire_dans(operateurs)
                
                tex.print("\\question $\\num{" .. nombre .. "}" .. op .. " " .. facteur .. " = $")
            end
        \end{luacode}
        \end{multicols}
    \exercice Poser les multiplications suivantes 
        \begin{multicols}{3}
        \begin{luacode}
            for i = 1,15
            do
                a = math.random(100,999) / math.pow(10, math.random(-1,3))
                b = math.random(100,999) / math.pow(10, math.random(-1,3))
                tex.print("\\question $\\num{" .. int(a) .. "} \\times \\num{" .. int(b) .. "} =$")
            end
        \end{luacode}
        \end{multicols}
    \exercice Convertir 
        \begin{multicols}{3}
        \begin{luacode}
            unites = {"\\metre", "\\gram", "\\second", "\\candela", "\\kelvin", "\\ampere", "\\mole"}
            prefixes = {"" , "\\kilo", "\\hecto", "\\deca", "\\deci", "\\centi", "\\milli"}

            for i = 1,15 
            do
                melanger(unites)
                melanger(prefixes)
                nombre = math.random(100,999) / math.pow(10, math.random(0,4))
                if nombre==math.floor(nombre) then nombre = math.floor(nombre) end
                tex.print("\\question $\\qty{" .. nombre .. "}{" .. prefixes[1] .. unites[1] .. "} = ~~~\\unit{" .. prefixes[2] .. unites[1] .. "}$") 
            end
        \end{luacode}
        \end{multicols}
\end{questions}

\end{document}