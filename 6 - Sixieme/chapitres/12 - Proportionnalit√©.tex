\documentclass[../Cours.tex]{subfiles}

\begin{document}
\chapitre{Proportionnalité}

\partie{Grandeurs proportionnelles}
\souspartie{Concept}

\definition{Une grandeur est une caractéristique quantifiable (souvent associée à une unité).}
\exemples{Distance (\unit{\metre}), Prix (\unit{\EURO}), Durée (\unit{\second}), Température (\unit{\degreeCelsius}), Masse (\unit{\kg}), Énergie (\unit{\joule}), Puissance (\unit{\watt}), Tension (\unit{\volt})}

\definition{Deux grandeurs sont proportionnelles si pour obtenir les valeurs de la première grandeur, on peut multiplier les valeurs de la deuxième grandeur \emph{par le même nombre}. Ce nombre est appelé \emph{coefficient de proportionnalité}.}

\begin{listedexemples}
    \item Le côté et le périmètre d'un carré sont \emph{proportionnels}.
    \item L'âge et la taille d'une personne \emph{ne sont pas proportionnels}.
\end{listedexemples}

\souspartie{Tableau de proportionnalité}

\propriete{Avec deux grandeurs proportionnelles, on peut construire un \emph{tableau de proportionnalité}.}

\exemple{
\begin{center}
    \begin{tabularx}{0.6\linewidth}{|p{5cm}|C|C|C|}\hline
        Nombre de pommes & 1 & 2 & 3 \\\hline
        Prix des pommes (\unit{\EURO}) & \num{0,30} & \num{0,60} & \num{0,90} \\\hline
    \end{tabularx}\\[1cm]
    \begin{tabularx}{0.6\linewidth}{|p{5cm}|C|C|C|}\hline
        Durée (\unit{\hour}) & 1 & 5 & 10 \\\hline
        Distance (\unit{\kilo\metre}) & 50 & 250 & 500 \\\hline
    \end{tabularx}
\end{center}
}

\souspartie{Représentation graphique}

\propriete{Deux grandeurs proportionnelles peuvent être représentées graphiquement par une droite passant par l'origine du repère.}

\exemple{
\begin{center}
    \begin{tabularx}{0.6\linewidth}{|p{5cm}|C|C|C|}\hline
        Nombre de pommes & 1 & 2 & 3 \\\hline
        Prix d'une pomme (\unit{\EURO}) & \num{0,30} & \num{0,60} & \num{0,90} \\\hline
    \end{tabularx}\\[1cm]\color{black}
    \begin{tikzpicture}
        \draw[-Latex] (0,0) -- (0,4);
        \draw[-Latex] (0,0) -- (4,0);
        \foreach \x in {1,...,3} {
            \draw (\x,-0.1) -- (\x,0.1);
            \node[below] at (\x,0) {$\x$};
        }
        \foreach \y in {0.1,0.2,0.3,0.4,0.5,0.6,0.7,0.8,0.9} {
            \draw (-0.1,{4*\y}) -- (0.1,{4*\y});
            \node[left,anchor=east] at (0,{4*\y}) {\tiny{\num{\y}}};
        }
        \draw[rouge,dashed] (1,0) -- (1,1.2) -- (0,1.2);
        \draw[rouge,dashed] (2,0) -- (2,2.4) -- (0,2.4);
        \draw[rouge,dashed] (3,0) -- (3,3.6) -- (0,3.6);
        \draw[bleu] (0,0) -- (3.1,{3.1*1.2});
    \end{tikzpicture}
\end{center}
}

\partie{Quatrième proportionnelle}
Dans cette partie, on liste plusieurs méthodes pour trouver la quatrième valeur manquante d'un tableau de proportionnalité :

\begin{center}
    \begin{tabularx}{0.6\linewidth}{|p{5cm}|C|C|C|}\hline
        Masse de raisins (\unit{\kg}) & 3 & 6 \\\hline
        Prix (\unit{\EURO}) & \num{7,80} & ? \\\hline
    \end{tabularx}
\end{center}

\souspartie{Homogénéité}

\methode{Multiplier (ou diviser) toute une colonne par \emph{la même valeur}.}

\begin{center}
    \begin{tikzpicture}
        \node at (0,0) {
            \begin{tabularx}{0.6\linewidth}{|p{5cm}|C|C|C|}\hline
                Masse de raisins (\unit{\kg}) & 3 & 6 \\\hline
                Prix (\unit{\EURO}) & \num{7,80} & ? \\\hline
            \end{tabularx}
        };
        \draw[-Latex] (2,0.8) arc (170:10:1 and 0.5);
        \draw[-Latex] (2,-0.8) arc (-170:-10:1 and 0.5);
        \node at (3,1.5) {$\times 2$};
        \node at (3,-1.5) {$\times 2$};
    \end{tikzpicture}
\end{center}

\souspartie{Passage à l'unité}

\methode{Se ramener à une colonne contenant le nombre 1.}

\begin{center}
    \begin{tikzpicture}
        \node at (0,0) {
            \begin{tabularx}{0.6\linewidth}{|p{5cm}|C|C|C|}\hline
                Masse de raisins (\unit{\kg}) & 3 & 1 & 6 \\\hline
                Prix (\unit{\EURO}) & \num{7,80} & ? & ? \\\hline
            \end{tabularx}
        };
        \draw[-Latex] (1,0.8) arc (170:10:0.7 and 0.5);
        \draw[-Latex] (1,-0.8) arc (-170:-10:0.7 and 0.5);
        \node at (1.7,1.5) {$\div 3$};
        \node at (1.7,-1.5) {$\div 3$};
        \draw[-Latex] (3,0.8) arc (170:10:0.7 and 0.5);
        \draw[-Latex] (3,-0.8) arc (-170:-10:0.7 and 0.5);
        \node at (3.7,1.5) {$\times 6$};
        \node at (3.7,-1.5) {$\times 6$};
    \end{tikzpicture}
\end{center}

\souspartie{Coefficient de proportionnalité}

\methode{Calculer le coefficient de proportionnalité, pour passer d'une ligne à l'autre.}

$$\dfrac{7,80}{3} = 2,60$$

\begin{center}
    \begin{tikzpicture}
        \node at (0,0) {
            \begin{tabularx}{0.6\linewidth}{|p{5cm}|C|C|}\hline
                Masse de raisins (\unit{\kg}) & 3 & 6 \\\hline
                Prix (\unit{\EURO}) & \num{7,80} & ? \\\hline
            \end{tabularx}
        };
        \draw[-Latex] (5.7,0.5) arc (90:-90:0.2 and 0.5);
        \node at (7.4,0) {$\times \qty{2,60}{\EURO\per\kg}$};
    \end{tikzpicture}
\end{center}

\souspartie{Additivité}

\methode{Additionner les contenus de deux colonnes.}

\begin{center}
    \begin{tikzpicture}
        \node at (0,0) {
            \begin{tabularx}{0.6\linewidth}{|p{5cm}|C|C|C|}\hline
                Masse de raisins (\unit{\kg}) & 3 & 6 & 9 \\\hline
                Prix (\unit{\EURO}) & \num{7,80} & \num{15,60} & ? \\\hline
            \end{tabularx}
        };
        \draw (1.8,1.1) circle (0.3);
        \draw (1,0.6) -- (1,1.1) -- (1.5,1.1);
        \draw (2.6,0.6) -- (2.6,1.1) -- (2.1,1.1);
        \node at (1.8,1.1) {$+$};
        \draw (1.8,1.4) -- (1.8,1.5) -- (4.5,1.5) -- (4.5,0.6);
        \draw (1.8,-1.1) circle (0.3);
        \draw (1,-0.6) -- (1,-1.1) -- (1.5,-1.1);
        \draw (2.6,-0.6) -- (2.6,-1.1) -- (2.1,-1.1);
        \node at (1.8,-1.1) {$+$};
        \draw (1.8,-1.4) -- (1.8,-1.5) -- (4.5,-1.5) -- (4.5,-0.6);
    \end{tikzpicture}
\end{center}

\souspartie{Produit en croix}

\methode{Multiplier deux valeurs en diagonale et diviser par une valeur latérale.}

\begin{center}
    \begin{tikzpicture}
        \node at (0,0) {
            \begin{tabularx}{0.6\linewidth}{|p{5cm}|C|C|C|}\hline
                Masse de raisins (\unit{\kg}) & 3 & 8 \\\hline
                Prix (\unit{\EURO}) & \num{7,80} & ? \\\hline
            \end{tabularx}
        };

        \node[anchor=west] at (6,0) {$\dfrac{\color{rouge}7,80 \times 8}{\color{vert}3} =~?$};
        \draw[very thick, red] (2.1,-0.35) -- (3.4,0.35);
        \draw[very thick, vert] (3.4,0.35) -- (2.1,0.35);
        \draw[very thick, -Latex] (2.1,0.35) -- (3.4,-0.35);
    \end{tikzpicture}
\end{center}

\partie{Pourcentages}

\definition{Un pourcentage est équivalent à une fraction dont le dénominateur vaut 100.}
\exemple{$\qty{27}{\%} = \frac{27}{100}$}

\paragraphe{rouge}{En particulier}{\vspace{-3ex}
    \begin{multicols}{2}
        \begin{itemize}
            \item $\qty{100}{\%}$ = 1
            \item $\qty{50}{\%} = \dfrac{1}{2}$
            \item $\qty{25}{\%} = \dfrac{1}{4}$
            \item $\qty{75}{\%} = \dfrac{3}{4}$
        \end{itemize}
    \end{multicols}
}

\exemple{}

\end{document}