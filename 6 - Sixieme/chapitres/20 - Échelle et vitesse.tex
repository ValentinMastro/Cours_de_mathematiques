\documentclass[../Cours.tex]{subfiles}

\begin{document}
\chapitre{Applications de la proportionnalité : échelles et vitesses}

\partie{Échelle}

\definition{L'échelle sur une carte est le coefficient de proportionnalité entre les distances réelles et les distances représentées sur la carte.}

\formule{$$\text{échelle} = \dfrac{\text{distance sur la carte}}{\text{distance réelle}}$$}

\exemple{Si une carte est à l'échelle $\frac{1}{2000}$, cela signifie que \qty{1}{cm} sur la carte représente \qty{2000}{cm} ($=\qty{20}{m}$) en réalité.\\Pour l'illustrer, construisons le tableau suivant :\\
\begin{center}
\begin{tabular}{|c|c|c}\hline
    Distance en réalité (en \unit{cm}) & \num{2000} & \hspace{3cm} \\\hline
    Distance sur la carte (en \unit{cm}) &  1 &  \\\hline
\end{tabular}%
\end{center}%
Sur cette même carte, si on mesure une distance de \qty{5,2}{cm}, à quelle distance réelle cela correspond-il ? (arrondir au \unit{\meter} si nécessaire). 
}

\remarque{Le numérateur et le dénominateur de l'échelle doivent être exprimés dans la même unité.}

\clearpage

\begin{questions}

\exercice\\ Sur un plan, un appartement est représenté par un rectangle de dimensions \qty{18}{cm} sur \qty{16,2}{cm}. La longueur réelle de l'appartement est de \qty{9}{m}.

\question Recopier et compléter le tableau suivant 
\begin{center}
\begin{tabular}{|c|c|c|c|}\hline
    Distance sur le plan (en \unit{cm}) & 18 & 1 & 16,2 \\\hline
    Distance dans la réalité (en \unit{cm}) & .... & .... & ....\\\hline
\end{tabular}
\end{center}

\question Écrire l'échelle de ce plan sous la forme $\dfrac{1}{....}$


\exercice\\ Sur un plan, un couloir de \qty{10}{m} de long est représenté par une longueur de \qty{20}{cm}.
\question Exprimer ces deux dimensions en \unit{cm}.
\question Déterminer l'échelle de ce plan

\exercice\\ Sur une carte au $\dfrac{1}{100000}$ la distance entre deux villes mesure \qty{9}{cm}.
\question Quelle est (en \unit{km}) la distance réelle entre ces deux villes ?

\exercice\\ On dispose d'une carte routière au $\dfrac{1}{1000000}$. \\Sachant que la distance entre Bourg-en-Bresse et Lyon est de \qty{62}{km}, quelle sera la distance \emph{sur la carte} entre ces deux villes ?

\end{questions}

\clearpage
\EXERCICES
\begin{questions}
    \exercice Une voiture parcoure \qty{302}{\kilo\metre} en \qty{2.5}{\hour}.
    \question Quelle est sa vitesse moyenne ?
    \question S'il fait beau et que l'on roule sur l'autoroute, enfreint-on la loi ?
    \question Et s'il pleut ?

    \exercice Une année-lumière est la distance que parcoure la lumière en 1 an. On la note $c \approx \qty{300000}{\kilo\metre\per\second}$.
    \question En rappelant que $\qty{1}{an} = \qty{365.25}{j}$, combien y a-t-il de secondes dans 1 an ?
    \question En déduire \qty{1}{al} = ?~\unit{\kilo\metre}.
    \question Proxima du Centaure est à \qty{4.246}{al} du Soleil. Donner cette distance en kilomètres.
\end{questions}


\end{document}