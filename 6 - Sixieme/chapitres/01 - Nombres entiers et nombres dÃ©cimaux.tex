\documentclass[../Cours.tex]{subfiles}

\color{bleu}

\begin{document}
\chapitre{Nombres entiers et nombres décimaux}

\partie{Nombres entiers}
\souspartie{Cardinal : principe d'énumération}



\souspartie{Notation positionnelle décimale}

\vocabulaire{Un chiffre est un symbole qui permet, en les assemblant, de construire des nombres.\\ Il en existe 10 dans le système dit << décimal >> : \\\centerline{0 ; 1 ; 2 ; 3 ; 4 ; 5 ; 6 ; 7 ; 8 ; 9}}

\vocabulaire{La position d'un chiffre dans un nombre détermine sa valeur, selon le tableau suivant :}







\partie{Nombres décimaux}

\end{document}