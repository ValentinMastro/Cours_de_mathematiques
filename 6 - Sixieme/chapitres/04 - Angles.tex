\documentclass[../Cours.tex]{subfiles}

\begin{document}
\chapitre{Angles}

\partie{Concept}

\definition{Un angle représente l'écartement entre deux demi-droites partageant la même origine.}

\illustration{%
\begin{center}
\begin{tikzpicture}
    \coordinate (O) at (0,0);
    \coordinate (A) at (2,0.5);
    \coordinate (B) at (2,-0.5);
    \draw ($(O)!1.5!(A)$) -- (O) -- ($(O)!1.5!(B)$);
    \draw[rouge,fill=rouge] ($(O)!0.5cm!(A)$) arc (15:-15:0.5) -- (O) -- cycle;
    \node at (A) {|};
    \node[above left] at (A) {$A$};
    \node at (B) {|};
    \node[below left] at (B) {$B$};
    \node[left] at (O) {$O$};
    \node[anchor=west] at (2.5,0) {\textcolor{rouge}{L'angle $\widehat{AOB}$}};
\end{tikzpicture}
\end{center}
}

\definition{%
\begin{itemize}
    \item Le \emph{sommet} de l'angle est l'origine commune.
    \item Les \emph{côtés} de l'angle sont les demi-droites.
\end{itemize}
}

\notation{Trois points avec un << chapeau >> \\ Le point du milieu est le sommet}

\end{document}