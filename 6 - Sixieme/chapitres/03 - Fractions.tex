\documentclass[../Cours.tex]{subfiles}

\begin{document}
\chapitre{Fractions}

\partie{Divisibilité}

\definition{ << Être divisible par >> signifie << faire partie de la table de multiplication de >>.}

\begin{listedexemples}
    \item $42$ est divisible par \textcolor{rouge}{6} car $42 = \textcolor{rouge}{6} \times 7$
    \item 35 est divisible par \textcolor{rouge}{5} car $35 = \textcolor{rouge}{5} \times 7$
\end{listedexemples}

\paragraphe{noir}{Critères de divisibilité}{
    \begin{tabularx}{\textwidth}{l|C|l}
        & règle & exemples \\\hline
        divisible par 2 & \makecell{le nombre est pair \\ le chiffre des unités est 0, 2, 4, 6 ou 8} & \makecell{\textcolor{vert}{102 est pair}\\ \textcolor{rouge}{113 est impair}} \\\hline
        divisible par 5 & le chiffre des unités est 0 ou 5 & \makecell{\textcolor{vert}{\num{1000000005}} \\ \textcolor{rouge}{22} } \\\hline
        divisible par 10 & le chiffre des unités est 0 & \makecell{ \textcolor{vert}{\num{4000000000}} \\ \textcolor{rouge}{13} } \\\hline
        divisible par 3 & la somme des chiffres est dans la table de 3 & \makecell{\textcolor{vert}{213 $\longrightarrow 2+1+3 = 6$}\\\textcolor{rouge}{$17 \longrightarrow 1+7 = 8$}} \\\hline
        divisible par 9 & la somme des chiffres est dans la table de 9 & \makecell{\textcolor{vert}{342 $\longrightarrow 3+4+2 = 9$}\\\textcolor{rouge}{$14 \longrightarrow 1+4 = 5$}} \\\hline
    \end{tabularx}
}

\end{document}