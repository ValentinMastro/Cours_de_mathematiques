\documentclass[../Cours.tex]{subfiles}

\begin{document}

\setcounter{chapitre}{-1}
\chapitre{Les nombres : vers la notation positionnelle}

\partie{Chiffres et nombres}

\definition{Il existe 10 \emph{chiffres} dans un système décimal : 0;1;2;3;4;5;6;7;8;9. En les combinant, on peut former des \emph{nombres}.}

\exemple{Avec les chiffres 1 et 2, je peux former les nombres 1 ; 2 ; 12 ; 21.}

\propriete{La valeur représentée par un chiffre dépend de sa position par rapport à la virgule.}

{
\small
\newcommand*\rot{\rotatebox{90}}
\begin{center}
\begin{tabularx}{\textwidth}{|*{13}{C|}}\hline
    \multicolumn{3}{|c}{Millions} & \multicolumn{3}{|c}{Milliers} & \multicolumn{3}{|c|}{Unités} & , & \multicolumn{3}{|c|}{Millièmes} \\\hline
    \rot{centaine de millions} & \rot{dizaine de millions} & \rot{millions} & \rot{centaine de milliers} & \rot{dizaine de milliers} & \rot{milliers} & \rot{centaines} & \rot{dizaines} & \rot{unités} & & \rot{dixièmes} & \rot{centièmes} & \rot{millièmes} \\\hline
\end{tabularx}
\end{center}
}

\begin{listedexemples}
    \item $45,219 = (4 \times 10) + (5 \times 1) + (2 \times 0,1) + (1 \times 0,01) + (9 \times 0,001)$
    \item $\num{17845,6} = 1 \times \num{10 000} + 7 \times \num{1 000} + 8 \times 100 + 4 \times 10 + 5 \times 1 + 6 \times 0,1$
    \item $\num{2,456} =$ \color{rouge} $ 2 \times 1 + \frac{4}{10} + \frac{5}{100} + \frac{6}{1000} =$ \color{vert} $ 2 \times 1 + 4 \times 0,1 + 5 \times 0,01 + 6 \times 0,001$ \color{bleu}
\end{listedexemples}

\partie{Comparaison de nombres}

\definition{Comparer deux nombres consiste à dire lequel est le plus grand ou le plus petit. On utilise les symboles $<$, $>$, $\infeg$, $\supeg$ et $=$.}

\begin{listedexemples}
    \item $45,5 < 45,68$
    \item $0,02 \supeg 0,01$
\end{listedexemples}

\clearpage \thispagestyle{empty}
\begin{questions}
    \enigme Ramsès a acheté des chameaux et dromadaires, tous normaux. Au total, il compte 21 bosses et 52 pattes. Il poste un soldat par chameau. De combien de soldats a-t-il besoin ?
    \enigme Par quelles opérations est-il possible d’obtenir 100 en utilisant 5 fois le nombre 5 ? 
\[ 5 ~?~ 5 ~?~ 5 ~?~ 5 ~?~ 5 = 100 \]
    \enigme Construis un carré et coupe-le en 7 carrés plus petits.
    \enigme Pierre dit à Marie : « J’ai découpé des petits rectangles de carton de 3 cm sur 5 cm » dans un grand rectangle de 15 cm sur 22 cm, sans qu’il n’y ait de chute ! " Marie réfléchit et lui répond : " Tu as raison, et je sais même combien tu en as découpé et comment tu les as découpés ! " Retrouver leur solution.
    \enigme Combien de secondes se sont écoulées depuis ta naissance ?
    \enigme Je viens de fêter ma milliardième seconde, quel est mon âge ?
    \enigme J’ai fait 10 km à vélo. Combien de tours a fait chacune des roues du vélo ?
\end{questions}

\clearpage \thispagestyle{empty}
\begin{questions}
    \enigme Ramsès a acheté des chameaux et dromadaires, tous normaux. Au total, il compte 21 bosses et 52 pattes. Il poste un soldat par chameau. De combien de soldats a-t-il besoin ?
    \enigme Par quelles opérations est-il possible d’obtenir 100 en utilisant 5 fois le nombre 5 ? 
\[ 5 ~?~ 5 ~?~ 5 ~?~ 5 ~?~ 5 = 100 \]
    \enigme Construis un carré et coupe-le en 7 carrés plus petits.
    \enigme Pierre dit à Marie : « J’ai découpé des petits rectangles de carton de 3 cm sur 5 cm » dans un grand rectangle de 15 cm sur 22 cm, sans qu’il n’y ait de chute ! " Marie réfléchit et lui répond : " Tu as raison, et je sais même combien tu en as découpé et comment tu les as découpés ! " Retrouver leur solution.
    \enigme Combien de secondes se sont écoulées depuis ta naissance ?
    \enigme Je viens de fêter ma milliardième seconde, quel est mon âge ?
    \enigme J’ai fait 10 km à vélo. Combien de tours a fait chacune des roues du vélo ?
\end{questions}

\clearpage
\begin{center}
    \begin{tikzpicture}
        \draw[-Latex] (0,0) -- (18,0);
        \foreach \x in {1,...,17} {
            \node at (\x,{0.3+0.4*mod(\x,2)}){\pgfmathparse{int((\x+4)*100)}\pgfmathresult};
            \draw (\x,-0.1) -- (\x,0.1);
        }
        \draw ({16.43-4},-0.5) -- ({16.43-4},-1.5);
        \draw ({17.15-4},-0.5) -- ({17.15-4},-1.5);
        \node[below] at (12.8,-1) {\rotatebox{90}{Louis XIV}};
    \end{tikzpicture}
\end{center}    

\end{document}