\documentclass[../Cours.tex]{subfiles}

\begin{document}
\chapitre{Symétrie axiale}

\partie{Tracer une figure symétrique}

\definition{On dit que deux figures $(F)$ et $(F')$ sont symétriques par rapport à une droite $(d)$ si, en pliant le long de la droite $(d)$, les figures $(F)$ et $(F')$ se superposent. La droite $(d)$ est appelée \emph{axe de symétrie}.}

\exemples{
    \begin{center}
        \begin{tikzpicture}[scale=0.4]
            \draw[gray] (0,0) grid (10,10);
            \draw[line width=0.5mm,rouge] (5,11) -- (5,-1);
            \draw[gray] (12,0) grid (22,10);
            \draw[line width=0.5mm,rouge] (11,5) -- (23,5);
            \draw[gray] (24,0) grid (34,10);
            \draw[line width=0.5mm,rouge] (23,-1) -- (35,11);
        \end{tikzpicture}
    \end{center}
}

\souspartie{Sans quadrillage}

\definition{Deux points $A$ et $A'$ sont symétriques par rapport à une droite $(d)$ si $(d)$ est la médiatrice du segment $[AA']$.}

\illustration{%
\begin{center}
    \begin{tikzpicture}
        \draw[fill=noir] (0,0) circle (0.05);
        \node[below right] at (0,0) {$A$};
        \draw (-3,-0.2) -- (2,1.5);
    \end{tikzpicture}
\end{center}
}

\partie{Tracer un axe de symétrie}


\definition{Pour tracer un axe de symétrie}


\clearpage
\begin{questions}
    \exercice Tracer le symétrique des points $A$, $B$, $C$ et $D$ par rapport à la droite $(d)$.

    \begin{center}
        \begin{tikzpicture}
            \draw[fill=noir] (0,0) circle (0.05) node[below] {$A$} -- (1,2) circle (0.05) node[right] {$B$} -- (-1,2,0) circle (0.05) node[left] {$C$} -- (-1.5,-1) circle (0.05) node[below] {$D$} -- (0,0);
            \draw (1.3,-2.5) -- (2.9,2.3);
            \node[right] at (1.3,-2.5) {$(d)$};
        \end{tikzpicture}
    \end{center}

    \exercice Tracer le symétrique de $(F)$ par rapport à la droite $(d)$ et le symétrique de $(G)$ par rapport à la droite $(e)$.

    \begin{center}
        \begin{tikzpicture}
            \draw (1,2) -- (2,0) -- (3,4) -- (5,2)) -- cycle;
            \draw (1.3,-2.5) -- (2.9,2.3);
            \node[right] at (1.3,-2.5) {$(d)$};
        \end{tikzpicture}
    \end{center}
\end{questions}

\end{document}

