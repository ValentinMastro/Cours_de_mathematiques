\documentclass[../Cours.tex]{subfiles}

\begin{document}
\clearpage
\thispagestyle{empty}

\titreDS

\begin{questions}
    \EXERCICE{4}
    Pierre va acheter 2 baguettes et 3 éclairs aux chocolats à la boulangerie.\\
    Une baguette coûte \qty{1,20}{\EURO} et un éclair au chocolat coûte \qty{2,20}{\EURO}.
        \question Quel calcul faut-il faire pour trouver le prix de l'ensemble de ce qu'il a acheté ?
        \question Faire le calcul, et conclure.

    \EXERCICE{4}
    Donner l'abscisse des points $A$, $B$, $C$, $D$

    \begin{center}
        \begin{tikzpicture}[scale=0.7]
            \draw[-latex] (0,0) -- (9,0);
            \foreach \x in {0,...,8} {
                \draw (\x,-0.1) -- (\x,0.1);
            }
            \node at (0,-0.4) {$0$};
            \node at (7,-0.4) {$7$};
            \node at (3,0.4) {$A$};
            \node at (5,0.4) {$B$};
        \end{tikzpicture}\hspace{1cm}
        \begin{tikzpicture}[scale=0.7]
            \draw[-latex] (0,-1) -- (9,-1);
            \foreach \x in {0,...,8} {
                \draw (\x,-1.1) -- (\x,-0.9);
            }
            \node at (0,-1.4) {$0$};
            \node at (7,-1.4) {$4$};
            \node at (2,-0.6) {$C$};
            \node at (8,-0.6) {$D$};
        \end{tikzpicture}
    \end{center}

    \EXERCICE{4} Reproduire le repère ci-dessous en y ajoutant les points suivants :

    \begin{center}
    \begin{tikzpicture}[scale=0.5]
        \draw[-latex] (0,0) -- (8,0);
        \draw[-latex] (0,0) -- (0,8);
        \foreach \x in {1,...,7} {
            \draw (\x,-0.1) -- (\x,0.1);
            \draw (-0.1,\x) -- (0.1,\x);
            \node[below] at (\x,0) {$\x$};
            \node[left] at (0,\x) {$\x$};
        };
        \node[below] at (0,0) {0};
    \end{tikzpicture}
    \end{center}

        \question 
            \subpart{$A(5;6)$, $B(5;2)$, $C(3;4)$, $D(7;4)$}
            \subpart{Sans justifier, de quelle forme est $ABCD$ ?}
        \question 
            \subpart{$E(1;2)$, $F(1;4)$, $G(2;7)$, $H(3;6)$, $I(2;2)$}
            \subpart{De quelle forme est $EFGHI$ ?}

    \EXERCICE{3}
        \question Donner deux nombres, entre 10 et 99, divisibles par 5.
        \question Donner deux nombres, entre 1000 et 9999, divisibles par 3.
        \question Donner deux nombres, entre 100 et 999, divisibles par 9.

    \EXERCICE{5}
    Dans cet exercice, \emph{on tracera tout sur la même figure.}
    \begin{multicols}{2}
        \question Tracer $[AB]$ : \qty{6}{\cm} et horizontal.
        \question Placer $H \in [AB]$ et $C \notin (AB) $
        \question Tracer $[HC)$ et $(AC]$
        \question Placer $I \in [HC]$ et $J \in [AB]$
        \question Tracer $(AB)$
    \end{multicols}

    \EXERCICE{(bonus) 1}
    Donner le 67ème chiffre après la virgule de $\dfrac{1234}{9999}$
\end{questions}

\end{document}