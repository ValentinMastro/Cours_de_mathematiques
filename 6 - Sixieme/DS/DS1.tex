\documentclass[addpoints,12pt]{exam}

\usepackage[locale=FR, group-digits=all, group-separator=\ , group-minimum-digits=4, per-mode = symbol]{siunitx}
\usepackage[margin=1.5cm]{geometry}
\usepackage{mdframed}
\usepackage{tikz}
\usepackage{frenchmath}
\DeclareSIUnit{\litre}{\ell}
\usepackage{amsmath}
\usepackage{eurosym}
\DeclareSIUnit{\EURO}{\text{\euro}}
\usepackage{wrapfig}
\usepackage{multicol}

\newcommand{\titre}[1]{%
    % #1 : numéro du DS
    \begin{center}%
        \Huge{DS n°#1}%    
    \end{center}%
}

\begin{document}

\titre{1}

\qformat{\textbf{\underline{Exercice \thequestion : }} \hfill [\thepoints]}
\bonusqformat{\textbf{\underline{Exercice \thequestion : }} \hfill [\thepoints]}
\begin{questions}
    \question[4]
    Pierre va acheter 2 baguettes et 3 éclairs aux chocolats à la boulangerie.\\
    Une baguette coûte \qty{1,20}{\EURO} et un éclair au chocolat coûte \qty{2,20}{\EURO}. 

    \begin{parts}
        \part Quel calcul faut-il faire pour trouver le prix de l'ensemble de ce qu'il a acheté ?
        \part Faire le calcul, et conclure.
    \end{parts}

    \question[4]
    Donner l'abscisse des points $A$, $B$, $C$, $D$

    \begin{center}
        \begin{tikzpicture}[scale=0.7]
            \draw[-latex] (0,0) -- (9,0);
            \foreach \x in {0,...,8} {
                \draw (\x,-0.1) -- (\x,0.1);
            }
            \node at (0,-0.4) {$0$};
            \node at (7,-0.4) {$7$};
            \node at (3,0.4) {$A$};
            \node at (5,0.4) {$B$};
        \end{tikzpicture}\hspace{1cm}
        \begin{tikzpicture}[scale=0.7]
            \draw[-latex] (0,-1) -- (9,-1);
            \foreach \x in {0,...,8} {
                \draw (\x,-1.1) -- (\x,-0.9);
            }
            \node at (0,-1.4) {$0$};
            \node at (7,-1.4) {$4$};
            \node at (2,-0.6) {$C$};
            \node at (8,-0.6) {$D$};
        \end{tikzpicture}
    \end{center}

    \question[4] Reproduire le repère ci-dessous en y ajoutant les points suivants :

    \begin{center}
    \begin{tikzpicture}[scale=0.5]
        \draw[-latex] (0,0) -- (8,0);
        \draw[-latex] (0,0) -- (0,8);
        \foreach \x in {1,...,7} {
            \draw (\x,-0.1) -- (\x,0.1);
            \draw (-0.1,\x) -- (0.1,\x);
            \node[below] at (\x,0) {$\x$};
            \node[left] at (0,\x) {$\x$};
        };
        \node[below] at (0,0) {0};
    \end{tikzpicture}
    \end{center}

    \begin{parts}
        \part 
        \begin{subparts}
            \subpart $A(5;6)$, $B(5;2)$, $C(3;4)$, $D(7;4)$
            \subpart Sans justifier, de quelle forme est $ABCD$ ?
        \end{subparts}
        \part 
        \begin{subparts}
            \subpart $E(1;2)$, $F(1;4)$, $G(2;7)$, $H(3;6)$, $I(2;2)$
            \subpart De quelle forme est $EFGHI$ ?
        \end{subparts}
    \end{parts}

    \question[3]
    \begin{parts}
        \part Donner deux nombres, entre 10 et 99, divisibles par 5.
        \part Donner deux nombres, entre 1000 et 9999, divisibles par 3.
        \part Donner deux nombres, entre 100 et 999, divisibles par 9.
    \end{parts}

    \question[5]
    Dans cet exercice, \textbf{on tracera tout sur la même figure.}
    \begin{multicols}{2}
    \begin{parts}
        \part Tracer $[AB]$ de longueur \qty{6}{\cm} et horizontal.
        \part Placer $H \in [AB]$ et $C \notin (AB) $
        \part Tracer $[HC)$ et $(AC]$
        \part Placer $I \in [HC]$ et $J \in [AB]$
        \part Tracer $(AB)$
    \end{parts}
    \end{multicols}

    \bonusquestion[1]
    Donner le 67ème chiffre après la virgule de $\dfrac{1234}{9999}$
\end{questions}

\end{document}