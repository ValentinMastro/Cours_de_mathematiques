\documentclass[12pt]{memoir}
\usepackage{importationsprojet}
\usepackage{setspace}


\fancypagestyle{plain}{
    \fancyhf{}
    \fancyhead[C]{
        \begin{tabular}{*{3}{|p{5.7cm}}|}\hline
            \large{Nom : .........................\phantom{É}} & \large{Prénom :} ......................... & \large{Classe :} .....................\\\hline
        \end{tabular}
    }
    \fancyfoot[R]{\textbf{Page~\thepage~sur~\pageref{LastPage}}}
    \renewcommand{\headrulewidth}{0pt}
    \renewcommand{\footrulewidth}{0pt}
}
\pagestyle{plain}

\begin{document}
\linespread{1.5}

\begin{tabular}{*{1}{|p{18cm}}|}\hline
    \normalsize Compétence travaillée : Convertir une longueur en utilisant les multiples et sous-multiples du mètre.\\\hline
\end{tabular}

\vspace{2cm}

\textbf{\underline{Cours :}}
\par L'unité légale de la longueur est le mètre.

\begin{center}
\begin{tabular}{|p{2.7cm}*{7}{|c}|}\hline
    \textit{Préfixe + unité} & \textbf{kilo}mètre & \textbf{hecto}mètre & \textbf{déca}mètre & mètre & \textbf{déci}mètre & \textbf{centi}mètre & \textbf{milli}mètre \\\hline
    \textit{Abréviation} & \unit{\kilo\meter} & \unit{\hecto\meter} & \unit{\deca\meter} & \unit{\meter} & \unit{\deci\meter} & \unit{\centi\meter} & \unit{\milli\meter}\\\hline
    & = \qty{1000}{\meter} & = \qty{100}{\meter} & = \qty{10}{\meter} & = \qty{1}{\meter} & = \qty{0.1}{\meter} & = \qty{0.01}{\meter} & = \qty{0.001}{\meter}\\\hline
\end{tabular}
\end{center}

\textit{\underline{Exemple 1 : }} \textsc{Convertir une longueur}\\[-0.2cm]

Pour convertir \qty{12,5}{hm} en \unit{cm}, j'écris que $\qty{12,5}{hm} = 12,5 \times \qty{1}{hm}$.\\
Puis je me demande combien fait \qty{1}{hm} en \unit{cm}.\\
Pour cela j'utilise le tableau, et je trouve que $\qty{1}{hm} = \qty{10000}{cm}$.\\
Je remplace dans mon calcul : $12,5 \times \qty{1}{m} = 12,5 \times \qty{10000}{cm} = \qty{125000}{cm}$.\\

\textit{\underline{Exemple 2 : }} \textsc{Comparer deux longueurs}\\[-0.2cm]

Règle : on ne peut comparer deux longueurs que \textbf{lorsqu'elles possèdent la même unité}.\\[-0.2cm]

Je veux comparer \qty{72}{cm} et \qty{5}{dm}.\\
J'ai un problème, les deux unités ne sont pas les mêmes.\\
Je vais convertir \qty{5}{dm} en \unit{cm}.\\
Je trouve que $\qty{5}{dm} = \qty{50}{cm}$.\\
Je dois donc comparer \qty{72}{cm} et \qty{50}{cm}.\\
Je trouve que $\qty{72}{cm} > \qty{50}{cm}$ donc $\qty{72}{cm} > \qty{5}{dm}$.\\


\textit{\underline{Exemple 3 : }} \textsc{Décomposer une longueur}\\[-0.2cm]

$\qty{12,345}{m} = \qty{10}{m} + \qty{2}{m} + \qty{0,3}{m} + \qty{0,04}{m} + \qty{0,005}{m}$\\
$\qty{12,345}{m} = \qty{1}{dam} + \qty{2}{m} + \qty{3}{dm} + \qty{4}{cm} + \qty{5}{mm}$\\

\clearpage

\newcommand{\dotss}{..................}
\newcommand{\dotsss}{............................................}

\begin{questions}
\setlength\itemsep{2em}

\exercice ~~~Convertir les longueurs suivantes\\
\begin{itemize}
\setlength\itemsep{1em}
    \item la hauteur de la tour Eiffel : ~~~$\qty{3,24}{hm}~= \dotsss~\unit{m}$ 
    \item la distance d'un marathon : ~~~$\qty{4212,5}{dam}~= \dotsss~\unit{km}$
    \item la taille d'une fourmi : ~~~$\qty{0,0025}{m}~= \dotsss~\unit{mm}$
\end{itemize}

\exercice ~~~Comparer les longueurs suivantes (avec les symboles $>, <$ ou $=$)\\
\begin{itemize}
\setlength\itemsep{1em}
    \item \qty{52}{hm} ............ \qty{5,2}{dam}
    \item \qty{7000}{m} ............ \qty{8}{km}
    \item \qty{9}{mm} ............ \qty{0,000 009}{km}
\end{itemize}

\exercice ~~~Décomposer les longueurs suivantes\\
\begin{itemize}
\setlength\itemsep{1em}
    \item $\qty{76,543}{hm} =~\dotss~\unit{km}~+~\dotss~\unit{hm}~+~\dotss~\unit{dam}~+~\dotss~\unit{m}~+~\dotss~\unit{dm}$
    \item $\qty{19,04}{dm} =~\dotss~\unit{m}~+~\dotss~\unit{dm}~+~\dotss~\unit{mm}$
    \item $\qty{90,034}{dam}~=~\dotss~\unit{hm}~+~\dotss~\unit{dm}~+~\dotss~\unit{cm}$
    \item $\qty{193,5}{cm} =~\dotsss\dotsss\dotsss$
    \item $\qty{934}{dm} =~\dotsss\dotsss\dotsss$
\end{itemize}

\exercice ~~~Tracer un triangle $ABC$ tel que\\
\begin{itemize}
\setlength\itemsep{1em}
    \item $AB = \qty{3}{cm}$
    \item $BC = \qty{35}{mm}$
    \item $AC = \qty{0,042}{m}$
\end{itemize}

\end{questions}

\end{document}