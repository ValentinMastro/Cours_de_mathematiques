\documentclass[12pt]{memoir}
\usepackage{importationsprojet}
\usepackage{setspace}


\fancypagestyle{plain}{
    \fancyhf{}
    \fancyhead[C]{
        \begin{tabular}{*{3}{|p{5.7cm}}|}\hline
            \large{Nom : .........................\phantom{É}} & \large{Prénom :} ......................... & \large{Classe :} .....................\\\hline
        \end{tabular}
    }
    \fancyfoot[R]{\textbf{Page~\thepage~sur~\pageref{LastPage}}}
    \renewcommand{\headrulewidth}{0pt}
    \renewcommand{\footrulewidth}{0pt}
}
\pagestyle{plain}

\begin{document}
\linespread{1.5}

\begin{tabular}{*{1}{|p{18cm}}|}\hline
    \normalsize Compétence travaillée : Calculer et convertir une durée ; calcul d'instant\\\hline
\end{tabular}

\vspace{2cm}

\textbf{\underline{Cours :}}
\par L'unité légale de la durée est la seconde.
\par Un jour est la durée que met la planète Terre pour faire un tour sur elle-même.
\par Une année est la durée que met la planète Terre pour faire une révolution autour du Soleil.

\begin{center}
\begin{tabular}{|p{2.5cm}*{5}{|c}|}\hline
    \textit{Unité} & année & jour & heure & minute & seconde \\\hline
    \textit{Abréviation} & an(s) & j & h & min & s \\\hline
    & = 365[,25] j & = \qty{24}{h} & = \qty{60}{min} & = \qty{60}{s} & \\\hline
\end{tabular}
\end{center}

\textit{\underline{Exemple 1 : }} \textsc{Lire l'heure}\\[-0.2cm]

\newcommand{\horloge}
{
    \begin{tikzpicture}
        \draw (0,0) circle (2);
        \draw[fill=black] (0,0) circle (0.05);
    \end{tikzpicture}
}

\textit{\underline{Exemple 2 : }} \textsc{Calculer une durée entre deux instants}\\[-0.2cm]

\clearpage

\newcommand{\dotss}{..................}
\newcommand{\dotsss}{............................................}

\begin{questions}
\setlength\itemsep{2em}

\exercice ~~~Calculer\\
\begin{itemize}
\setlength\itemsep{1em}
    \item 4 semaines = \dotss jours
    \item 3 jours = \dotss heures
    \item 10 heures = \dotss minutes
    \item 1 heure = \dotss secondes
\end{itemize}

\exercice ~~~Compléter les horloges avec les heures correspondantes\\
\begin{itemize}
\setlength\itemsep{1em}
    \item 10h10
    \item << 9 heures et quart >>
    \item << 7 heures moins le quart >>
    \item 20h50
\end{itemize}

\end{questions}


\end{document}