\documentclass[11pt,addpoints]{exam}

\usepackage{tikz}
\usepackage{tkz-tab}
\usepackage[margin=2cm]{geometry}
\usepackage{amsmath,amssymb}

\newcommand{\N}{\mathbb{N}}

\begin{document}

\begin{center}
    \huge{Interrogation sur le comportement des suites numériques}
\end{center}

\vfill

\qformat{\textbf{Exercice \thequestion)}~\hfill}
\begin{questions}

\question[2] Recopier et compléter les phrases suivantes
\begin{parts}
    \part << Si une suite $(u_n)_{_{n\in\N}}$ est croissante, alors pour tout $n\in\N$, $u_{n+1}-u_n$......... >>
    \part << Si une suite $(v_n)_{_{n\in\N}}$ est décroissante, et que tous ses termes sont ............., alors pour tout $n\in\N$, $\frac{v_{n+1}}{v_n}$......... >>
    \part << Si une suite $(w_n)_{_{n\in\N}}$ est croissante, et qu'il existe une fonction $f$ définie sur $[0;+\infty[$ telle que pour tout $n\in\N$, $w_n = f(n)$ alors $f$ est .............. sur $[0;
    +\infty[$ >>
    \part << Si une suite $(t_n)_{_{n\in\N}}$ est constante, alors pour tout $n\in\N$, $t_{n+1}-t_n$......... >>
\end{parts}

\question[10] Donner les variations des suites suivantes
\begin{parts}
    \part Pour tout $n\in\N$, $u_n = 2n+5$
    \part Pour tout $n\in\N$, $v_n = 0{,}7^n$
    \part Pour tout $n\in\N$, $w_n = 3\sqrt{n}$
    \part Pour tout $n\in\N$, $x_{n+1} = x_n - n^2$
    \part Pour tout $n\in\N$, $y_{n+1} = y_n \times n$
    \part Pour tout $n\in\N$, $z_n = -2\times 3^n$
\end{parts}

\question[2] Conjecturer les limites des suites suivantes.
\begin{center}
    \begin{tikzpicture}[scale=0.1]
        \draw[black,-latex] (0,0) -- (60,0);
        \draw[black,-latex] (0,0) -- (0,40);
        \foreach \x in {0,10,...,60}
        {
            \node[black,below] at (\x,0) {\small{\x}};
        }
        \foreach \n in {1,...,60}
        {
            \draw (\n,{(30*\n)/(\n+3)}) circle (0.15);
        }
        \foreach \y in {1,...,4} 
        {
            \node[black,left] at (0,10*\y) {\small{\y}};
        }
        \draw[black,dashed] (0,30) -- (60,30);
    \end{tikzpicture}\hfill
    \begin{tikzpicture}[scale=0.1]
        \draw[black,-latex] (0,0) -- (40,0);
        \draw[black,-latex] (0,0) -- (0,40);
        \foreach \x in {0,10,...,40}
        {
            \node[black,below] at (\x,0) {\small{\x}};
            \node[black,left] at (0,\x) {\small{\x}};
        }
        \foreach \n in {1,...,40}
        {
            \draw (\n,\n^2/50) circle (0.15);
        }
    \end{tikzpicture}
\end{center}

\question[6] Soit la suite $(a_n)_{_{n\in\N}}$ définie pour tout $n\in\N$ par $a_n = 2n^2+5n+4$.
\begin{parts}
    \part De trois manières différentes, déterminer les variations de la suite $(a_n)$.
    \part Que peut-on conjecturer quant à la limite de la suite $(a_n)$ ?
\end{parts}


\vfill

\hqword{Exercice}
\hpword{Barème}
\hsword{Points obtenus}

\begin{center}
\gradetable[h][questions]
\end{center}

\end{questions}
\end{document}