\documentclass[a4paper, 11pt, oneside]{memoir}

\usepackage{importationsprojet}

\pagestyle{plain}

\begin{document}

\titre{première spé maths}

\setcounter{tocdepth}{3}
\tableofcontents*

\setcounter{chapitre}{6}
\ajouterChapitre{chapitres/07 - Comportement des suites numeriques.tex}

\clearpage
\begin{questions}

\exercice Étudier les variations de trois manières différentes et conjecturer la limite (avec l'aide de la calculatrice) de la suite $(u_n)$ définie par $u_n = 3n+7$
\exercice Étudier les variations d'au moins une manière et conjecturer la limite (avec l'aide de la calculatrice) de la suite $(v_n)$ définie par $v_n = 3\sqrt{n}$
\exercice Étudier les variations d'au moins deux manières et conjecturer la limite (avec l'aide de la calculatrice) de la suite $(w_n)$ définie par $w_n = \dfrac{2n^2+2}{3n^2+3}$
\exercice Étudier les variations et conjecturer la limite (avec l'aide de la calculatrice) de la suite $(x_n)$ définie par $x_n = 3$ et $x_{n+1} = x_n + 2n^2+2$


\end{questions}


\end{document}
