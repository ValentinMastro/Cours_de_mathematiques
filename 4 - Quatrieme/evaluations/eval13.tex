\documentclass{../../vmqcm}
\VMQCMgraine{123456789}
\setlength{\VMQCMLargeurEnonce}{0.23\textwidth}

%%%%%%%%%%%%%%%
%% QUESTIONS %%
%%%%%%%%%%%%%%%

\directlua{eval = require("eval13.lua")}

\newcommand{\questionA}{%
    \directlua{eval.qA()}
}

\newcommand{\questionB}{
    \directlua{eval.qB()}
}

\newcommand{\questionC}{
    \directlua{eval.qC()}
}

\newcommand{\questionD}{
    \directlua{eval.qD()}
}

\newcommand{\questionE}{
    \directlua{eval.qE()}
}

\newcommand{\questionF}{
    \directlua{eval.qF()}
}

\newcommand{\questionG}{
    \directlua{eval.qG()}
}

\newcommand{\questionH}{
    \directlua{eval.qH()}
}

\newcommand{\questionI}{
    \directlua{eval.qI()}
}

\newcommand{\questionJ}{
    \directlua{eval.qJ()}
}

\newcommand{\questionK}{
    \directlua{eval.qK()}
}

\setcounter{VMQCMNumeroEvaluation}{13}
\renewcommand{\VMQCMCalculatrice}{Calculatrice autorisée}
\begin{document}

\foreach \n in {1,...,1}{%
\begin{VMQCM}{Évaluation n°\theVMQCMNumeroEvaluation{}~-~4ème}
    \questionG
    \questionA
    \questionB
    \questionI
    \questionC
    \questionH
    \questionJ
    \questionG
    \questionI
    \questionA
    \questionD
    \questionK
    \questionH
    \questionK
    \questionF
    \questionJ
    \questionA
    \questionE
    \questionF
    \questionB
\end{VMQCM}%
}

\end{document}