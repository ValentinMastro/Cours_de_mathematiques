\documentclass{../../vmdm}

\begin{document}
\boxedpoints

\VMDMTitre{octobre 2022 (4ème)}
\vspace{0.2cm}
\VMDMRegles{}
\vspace{0.2cm}

\qformat{Exercice \thequestion \dotfill (\totalpoints~points)}
\begin{questions}
    \question[4]
    \begin{parts}
        \part
        \begin{subparts}
            \subpart Calculer $1 + 8 \times 1$ \hfill \rep{9}
            \subpart Calculer $2 + 8 \times 12$ \hfill\rep{98}
            \subpart Calculer $3 + 8 \times 123$ \hfill\rep{987}
        \end{subparts}
        \part Quel calcul permet d'obtenir \num{987654321} ? Vérifier.
    \end{parts}
    \question[8]
    \begin{parts}
        \part Dessiner en perspective cavalière un prisme droit à base hendécagonale
        \part Dessiner en perspective cavalière un prisme droit à base heptagonale
        \part Combien de faces a un prisme droit à base ennéadécagonale ? \hfill\rep{21}
    \end{parts}
    \question[8]
    On considère un cube $ABCDEFGH$ d'arête \qty{1}{\centi\metre} comme suit :
    \begin{center}
        \begin{tikzpicture}
            \tikzmath{ \r=2; }
            \coordinate (A) at (0,0);
            \coordinate (B) at (\r,0);
            \coordinate (C) at (\r,\r);
            \coordinate (D) at (0,\r);
            \coordinate (u) at (0.7,0.7);
            \coordinate (E) at ($(A)+(u)$);
            \coordinate (F) at ($(B)+(u)$);
            \coordinate (G) at ($(C)+(u)$);
            \coordinate (H) at ($(D)+(u)$);
            \foreach \p in {A,D,E,H} {
                \node[left] at (\p) {$\p$};
            }
            \foreach \p in {B,C,F,G} {
                \node[right] at (\p) {$\p$};
            }
            \draw (A) -- (B) -- (C) -- (D) -- cycle;
            \draw (D) -- (H) -- (G) -- (F) -- (B) (G) -- (C);
            \draw[dashed] (A) -- (E) -- (H) (E) -- (F);
        \end{tikzpicture}
    \end{center}
    \begin{parts}
        \part Montrer que $DG = \sqrt{2}~\unit{\cm}$
        \part Montrer que $DF = \sqrt{3}~\unit{\cm}$ \hfill (indice : considérer le triangle $DGF$)
    \end{parts}
\end{questions}

\vfill

\begin{center}
\gradetable[h][questions]
\end{center}

\end{document}