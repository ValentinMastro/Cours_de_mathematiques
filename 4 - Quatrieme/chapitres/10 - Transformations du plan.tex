\documentclass["../Cours.tex"]{subfiles}

\begin{document}
\chapitre{Transformations du plan}

\partie{Symétrie axiale}

\definition{Deux figures sont symétriques par rapport à une droite si, en <<repliant>> par rapport à la droite, les deux figures se superposent.}

\illustration{
    \begin{center}
        \begin{tikzpicture}
            \draw (2,0) -- +(3,0) -- +(0,2) -- cycle;
            \draw[xscale=-1] (2,0) -- +(3,0) -- +(0,2) -- cycle;
            \draw (0,3) -- (0,-1);
        \end{tikzpicture}
    \end{center}    
}

\newcommand{\bleu}[1]{\textcolor{bleu}{#1}}

\propriete{La symétrie axiale est une isométrie :
\begin{itemize}
    \item conservation de la nature : \bleu{(le symétrique d'un cercle est un cercle)}
    \item conservation des longueurs : \bleu{(le symétrique d'un segment de longueur \qty{2}{\centi\metre} fait \qty{2}{\centi\metre})}
    \item conservation des angles : \bleu{(le symétrique d'un angle de \ang{30} fait \ang{30})}
\end{itemize}
}

\exemple{$AB=\qty{6}{\centi\metre}$, $AC=\qty{5}{\centi\metre}$, $BC=\qty{4}{\centi\metre}$\\
Tracer le symétrique du point C par rapport à la droite $(AB)$.
}

\propriete{$C'$ est le symétrique de $C$ par rapport à $(AB) \Longleftrightarrow (AB)$ est la médiatrice de $[CC']$.}

\partie{Symétrie centrale}

\definition{Deux figures sont symétriques par rapport à un point (centre de symétrie) lorsqu'en faisant tourner la première figure autour du centre, on obtient la deuxième figure au bout d'un demi-tour.}

\propriete{$A'$ est le symétrique du point $A$ par rapport à $O \Longleftrightarrow OA=OA'$ et O, A, $A'$ sont alignés.}

\clearpage
\EXERCICES
\begin{questions}
    \exercice $ABCD$ est un parallélogramme.\\
    $O$ est l'intersection des diagonales de $ABCD$.
        \question Tracer la figure avec $AB=\qty{5}{\centi\metre}$, $BC=\qty{3}{\centi\metre}$ et $\widehat{ABC}=\ang{60}$.
        \question Démontrer que $A$ est le symétrique de $C$ par rapport à $O$.
        \question Sans justifier, donner le symétrique de $ABC$ par rapport à $O$.
    \exercice $[AB]$ est un segment de longueur \qty{1}{\centi\metre}.
    \begin{itemize}
        \item $A_1$ est le symétrique de $A$ par rapport à $B$.
        \item $A_2$ est le symétrique de $A$ par rapport à $A_1$.
        \item $A_3$ est le symétrique de $A$ par rapport à $A_2$.
        \item $A_4$ est le symétrique de $A$ par rapport à $A_3$.
    \end{itemize}
        \question Quelle est la longueur de $[AA_4]$ ?
        \question On réitère la construction avec $A_5$, $A_6$, $A_7$, etc. Trouver le plus petit $n$ tel que $AA_n > \qty{100}{\metre}$.
\end{questions}

\end{document}