\documentclass["../Cours.tex"]{subfiles}

\begin{document}
\chapitre{Équation}

\clearpage
\EXERCICES
\begin{questions}
    \exercice 
    \question L'équation $8x-2=5x-6$ admet pour solution :
        \textbf{a)} $x=-\dfrac{4}{3}$~
        \textbf{b)} $x=\dfrac{4}{3}$~
        \textbf{c)} $x=-\dfrac{3}{4}$~
    \question L'équation $7x+2=4x+7$ admet pour solution :
        \textbf{a)} $x=-\dfrac{5}{3}$~
        \textbf{b)} $x=\dfrac{5}{3}$~
        \textbf{c)} $x=-\dfrac{3}{5}$~
    \exercice Une équation qui a la même solution de l'équation $3y-6=9y-9$ est :\\
    \centerline{\hfill \textbf{a)} $5t=3$ \hfill \textbf{b)} $2t-1=0$ \hfill \textbf{c)} $3t-4=2$ \hfill}
\end{questions}

\clearpage
\CORRECTIONS
\begin{questions}
    \exercice 
    \question 
    \begin{align*}
        8x-2 &= 5x-6  & \mbox{il y a un $5x$ à droite, je veux l'enlever}\\
        8x-2\textcolor{rouge}{~-~5x} &= 5x-6\textcolor{rouge}{~-~5x} & \mbox{pour cela, je retire $5x$ de chaque côté}\\
        3x-2 &= -6  & 8x-5x=3x\\
        3x-2\textcolor{vert}{~+~2} &= -6\textcolor{vert}{~+~2} & \mbox{pour me débarrasser du $-2$ à gauche, j'ajoute $+2$}\\
        3x &= -4 & -6+2=-4 \\
        \dfrac{3x}{3} &= \dfrac{-4}{3} & \mbox{j'ai $3x$, mais je veux juste $x$, je vais donc diviser par 3} \\
        x &= \fbox{$-\dfrac{4}{3}$}
    \end{align*}
    \question 
    \begin{align*}
        7x+2 &= 4x+7  & \mbox{il y a un $4x$ à droite, je veux l'enlever}\\
        7x+2\textcolor{rouge}{~-~4x} &= 4x+7\textcolor{rouge}{~-~4x} & \mbox{pour cela, je retire $4x$ de chaque côté}\\
        3x+2 &= +7  & 7x-4x=3x\\
        3x+2\textcolor{vert}{~-~2} &= +7\textcolor{vert}{~-~2} & \mbox{pour me débarrasser du $+2$ à gauche, j'ajoute $-2$}\\
        3x &= +5 & +7-2=+5 \\
        \dfrac{3x}{3} &= \dfrac{5}{3} & \mbox{j'ai $3x$, mais je veux juste $x$, je vais donc diviser par 3} \\
        x &= \fbox{$\dfrac{5}{3}$}
    \end{align*}

    \exercice Dans l'énoncé, on a l'équation $3y-6=9y-9$. Résolvons-la.
    \begin{align*}
        3y-6 &= 9y-9  & \mbox{il y a un $9y$ à droite, je veux l'enlever}\\
        3y-6\textcolor{rouge}{~-~9y} &= 9y-9\textcolor{rouge}{~-~9y} & \mbox{pour cela, je retire $9y$ de chaque côté}\\
        -6y-6 &= -9  & 3y-9y=-6y\\
        -6y-6\textcolor{vert}{~+~6} &= -9\textcolor{vert}{~+~6} & \mbox{pour me débarrasser du $-6$ à gauche, j'ajoute $+6$}\\
        -6y &= -3 & -9+6=-3 \\
        \dfrac{-6y}{6} &= \dfrac{-3}{-6} & \mbox{j'ai $-6y$, mais je veux juste $y$, je vais donc diviser par -6} \\
        y &= \dfrac{-3}{-6} = \dfrac{3}{6} = \fbox{$\dfrac{1}{2}$} & \mbox{je simplifie la fraction}
    \end{align*}

    \clearpage

    On dit dans l'énoncé qu'une des trois équations a la même solution. On va donc tester les trois équations avec la solution qu'on a trouvé, c'est-à-dire $\frac{1}{2}$ :\\

    (*)~ Pour la première $5t=3$, on va calculer $5t$ avec notre solution et voir si cela donne bien 3 ou non.\\
    En l'occurrence, $5t = 5\times \frac{1}{2} = \frac{5}{2} = 2{,}5$. Ca ne fait donc pas 3.\\

    (*) Pour la deuxième $2t-1=0$, on va calculer $2t+1$ avec notre solution et voir si cela donne 0.\\
    En l'occurrence, $2t-1 = 2 \times \frac{1}{2} - 1 = 1 - 1 = 0$. C'est ce que l'on voulait trouver. Donc pour cette équation, $\frac{1}{2}$ est bien une solution.\\

    (*) Pour la troisième, $3t-4=2$, on va calculer $3t-4$ avec notre solution et voir si cela donne 2.\\
    En l'occurrence, $3t-4 = 3 \times \frac{1}{2} - 4 = \frac{3}{2} - 4 = -2{,}5$. Ca ne fait donc pas 2.
\end{questions}


\end{document}