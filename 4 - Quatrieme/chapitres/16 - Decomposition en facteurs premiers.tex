\documentclass["../Cours.tex"]{subfiles}

\begin{document}
\chapitre{Décomposition en facteurs premiers}


\partie{Nombres premiers}
\definition{On dit que $a$ est un diviseur de $b$ si le reste de la division euclidienne de $b$ par $a$ vaut 0.}

\exemple{3 est un diviseur de 15. Autrement dit, 15 est dans la table de 3.}

\definition{Un nombre premier est un entier $\neq 1$ ayant pour seuls diviseurs 1 et lui-même.}

\exemples{2;3;5;7;11;13;17;19;23;29;...}

\partie{Théorème fondamental de l'arithmétique}

\theoreme{}{Pour n'importe quel nombre entier $\supeg 2$, il existe une \underline{unique} façon de la décomposer en produits de facteurs premiers.}

\begin{listedexemples}
    \item 
    \begin{align*}
        219 &= 3 \times 73
    \end{align*}
    \item 
    \begin{align*}
        110 &= 10 \times 11 \\
        &= 2 \times 5 \times 11 \\
    \end{align*}
    \item 
    \begin{align*}
        625 &= 5 \times 125 \\
        &= 5 \times 5 \times 25 \\
        &= 5 \times 5 \times 5 \times 5 \\
        &= 5^4
    \end{align*}
\end{listedexemples}


\clearpage
\EXERCICES
\begin{questions}
    \exercice Décomposer en facteurs premiers les nombres suivants
    \begin{multicols}{3}
        \questionX 95
        \questionX 105
        \questionX 344
        \questionX 72
        \questionX 100
        \questionX 87
        \questionX 66
        \questionX 3590
        \questionX 144
    \end{multicols}
\end{questions}

\end{document}