\documentclass["../Cours.tex"]{subfiles}

\begin{document}

\begin{questions}

    \exercice Dans les deux textes, souligner en rouge les verbes conjugués au passé composé et en vert les verbes conjugués à l’imparfait.\\

    \underline{Texte 1 :} C’était l’hiver. Il neigeait. Il faisait presque nuit, la route était glissante. J’étais pressé, je marchais vite. J’ai glissé, je suis tombé sur le bord d’un trottoir et je me suis cassé le nez et une dent.\\

    
    \underline{Texte 2 :} Quand j’étais petite, j’allais à l’école à bicyclette, je parcourais \qty{3}{\kilo\metre} tous les matins et tous les soirs. À 16 ans, je suis entrée au lycée et mon père m’a offert un scooter.


    \exercice Conjuguer chaque verbe aux huit temps de l'indicatif (présent, passé composé, passé simple, passé antérieur, futur, futur antérieur, imparfait, plus-que-parfait)
    \question recycler

    \begin{center}
        \begin{tabularx}{\linewidth}{|l|X|}\hline
            Présent & Je recycle, Tu recycles, Il recycle, Nous recyclons, Vous recyclez, Ils recyclent \\\hline
            Passé composé & \\\hline
            Passé simple & \\\hline
            Passé antérieur & \\\hline
            Futur & \\\hline
            Futur antérieur & \\\hline
            Imparfait & \\\hline
            Plus-que-parfait & \\\hline
        \end{tabularx}
    \end{center}
    
    \question vendre
    \begin{center}
        \begin{tabularx}{\linewidth}{|l|X|}\hline
            Présent &  \\\hline
            Passé composé & \\\hline
            Passé simple & \\\hline
            Passé antérieur & \\\hline
            Futur & \\\hline
            Futur antérieur & \\\hline
            Imparfait & \\\hline
            Plus-que-parfait & \\\hline
        \end{tabularx}
    \end{center}
    
    \question s'habiller
    \begin{center}
        \begin{tabularx}{\linewidth}{|l|X|}\hline
            Présent &  \\\hline
            Passé composé & \\\hline
            Passé simple & \\\hline
            Passé antérieur & \\\hline
            Futur & \\\hline
            Futur antérieur & \\\hline
            Imparfait & \\\hline
            Plus-que-parfait & \\\hline
        \end{tabularx}
    \end{center}

    \clearpage
    \exercice Réécrire au passé composé les vers suivants
    \begin{center}
        Ici, sous de grands toits où scintille le verre,\\
        La vapeur se condense en force prisonnière :\\
        Des mâchoires d'acier mordent et fument ;\\
        De grands marteaux monumentaux\\
        Broient des blocs d'or sur des enclumes,\\
        Et, dans un coin, s'illuminent les fontes\\
        En brasiers tors et effrénés qu'on dompte. \\
    \end{center}

    \exercice Réécrire ce texte au futur. Remplacer les temps simples par le futur et les temps composés par le futur antérieur.

    << Chaque année, à la fête de la Saint-Pierre, s'ouvre « la foire aux fiancés ». Ce jour-là, il y a réunion de toutes les filles du comitat. Elles sont venues avec leurs plus belles carrioles attelées de leurs meilleurs chevaux ; elles ont apporté leur dot, c'est-à-dire des vêtements filés, cousus, brodés de leurs mains, enfermés dans des coffres aux brillantes couleurs ; familles, amies, voisines, les ont accompagnées. Et alors arrivent les jeunes gens, parés de superbes habits, ceints d'écharpes de soie. Ils courent la foire en se pavanant ; ils choisissent la fille qui leur plaît ; ils lui remettent un anneau et un mouchoir en signe de fiançailles, et les mariages se font au retour de la fête. >>

    \exercice Réécrivez ce texte au conditionnel présent.

    << Un pauvre homme passait dans le givre et le vent.\\
    Je cognai sur ma vitre ; il s'arrêta devant\\
    Ma porte, que j'ouvris d'une façon civile.\\
    Les ânes revenaient du marché de la ville,\\
    Portant les paysans accroupis sur leurs bâts.\\
    C'était le vieux qui vit dans une niche au bas\\
    De la montée, et rêve, attendant, solitaire,\\
    Un rayon du ciel triste, un liard de la terre,\\
    $\left[...\right]$ Il s'approcha du feu.\\
    Son manteau, tout mangé des vers, et jadis bleu,\\
    Étalé largement sur la chaude fournaise,\\
    Piqué de mille trous par la lueur de braise,\\
    Couvrait l'âtre, et semblait un ciel noir étoilé.\\
    Et, pendant qu'il séchait ce haillon désolé\\
    D'où ruisselait la pluie et l'eau des fondrières,\\
    Je songeais que cet homme était plein de prières,\\
    Et je regardais, sourd à ce que nous disions,\\
    Sa bure où je voyais des constellations. >> \\

    \exercice Réécrivez ce texte au présent de l'indicatif.

    << Le vacarme des marteaux de bois assourdissait le camp Lannister, où l'on s'affairait à construire une nouvelle tour de siège. Il y en avait déjà deux d'achevées, à moitié recouvertes de cuir cru de cheval. Entre elles reposait un bélier roulant fait d'un tronc d'arbre à la pointe durcie au feu et qui, suspendu par des chaînes à son affût, bénéficiait d'une toiture de bois. >>


\end{questions}

\end{document}