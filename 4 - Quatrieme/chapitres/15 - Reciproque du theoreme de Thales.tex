\documentclass["../Cours.tex"]{subfiles}

\begin{document}
\chapitre{Réciproque du théorème de Thalès}

\partie{Réciproque}

\paragraphe{rouge}{\textsc{Réciproque du théorème de Thalès}}{
Si : 
\begin{itemize}
    \item les points A, B et D sont alignés
    \item les points A, C et E sont alignés
    \item $\dfrac{AD}{AB} = \dfrac{AE}{AC} = \dfrac{DE}{BC}$ \hfill \textcolor{noir}{(NB : 2 fractions égales suffisent)}
\end{itemize}
alors :
$$(BC) \paral (DE)$$
}

\clearpage

\begin{redaction}{
\begin{center}
\begin{tikzpicture}[scale=0.8]
    \draw (0,0) node[left] {$O$} -- (9,3) node[above left] {$M$};
    \draw (0,0) -- (9,-3) node[below left] {$N$};
    \draw (4.5,1.5) node[above]{$L$} -- (4.5,-1.5) node[below]{$K$};
    \draw (8.4,2.8) -- (8.4,-2.8);
    \node[anchor=west] at (13,3) {$OM=12$};
    \node[anchor=west] at (13,2) {$OL=3$};
    \node[anchor=west] at (13,1) {$MN=24$};
    \node[anchor=west] at (13,0) {$LK=6$};
    \node[anchor=west] at (13,-3) {$\Rightarrow$ Montrer que $(LK) \paral (MN)$};
\end{tikzpicture}
\end{center}
}
    
\end{redaction}



\end{document}