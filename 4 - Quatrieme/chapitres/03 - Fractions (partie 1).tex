\documentclass["../Cours.tex"]{subfiles}

\begin{document}
\chapitre{Fractions et proportionnalité}

\partie{Calculs fractionnaires}

\definition{Soient deux nombres $a$ et $b$, avec $b \neq 0$. Le quotient de $a$ par $b$ est le nombre qui, lorsqu'il est multiplié par $b$, donne $a$. $$\dfrac{a}{b} \times b = a$$}

\notation{On le note $\dfrac{a}{b}$. (écriture fractionnaire)}

\begin{listedexemples}
    \item $\dfrac{3}{4} \times 4 = \dfrac{3}{\cancel{4}} \times \cancel{4} = 3$
    \item $\dfrac{7,2}{5,8} \times 5,8 = \dfrac{7,2}{\cancel{5,8}} \times \cancel{5,8} = 7,2$
\end{listedexemples}


\propriete{Soit deux nombres $a$ et $b$ avec $b \neq 0$. Soit $k$ avec $k \neq 0$. 
$$\dfrac{a}{b} = \dfrac{ka}{kb}$$
}

\begin{listedexemples}
    \item $\dfrac{16}{20} = \dfrac{16 \times 4}{20 \times 4} = \dfrac{64}{80}$
    \item $\dfrac{35}{75} = \dfrac{35 \div 5}{75 \div 5} = \dfrac{7}{15}$
\end{listedexemples}

\definition{Simplifier une fraction, c'est trouver une autre fraction qui lui est égale avec un numérateur et un dénominateur plus petits.}

\propriete{Pour additionner deux fractions, il faut qu'elles aient le même dénominateur.}

\propriete{Soient deux nombres $a$ et $b$, $b \neq 0$ et deux nombres $c$ et $d$, $d \neq 0$. $$ \dfrac{a}{b} + \dfrac{c}{d} = \dfrac{ad+cb}{bd}$$}

\begin{listedexemples}
    \item $\dfrac{3}{7} + \dfrac{12}{8} = \dfrac{3 \times 8}{7 \times 8} + \dfrac{12 \times 7}{8 \times 7}$ 
\end{listedexemples}

\propriete{Soient $a$ et $b$ deux nombres, $b \neq 0$.\\ Soient $c$ et $d$ deux nombres, $d \neq 0$. $$\dfrac{a}{b} \times \dfrac{c}{d} = \dfrac{a \times c}{b \times d}$$}

\exemple{$$\dfrac{7}{5} \times \dfrac{3}{4} = \dfrac{7 \times 3}{5 \times 4} = \dfrac{21}{20}$$}

\propriete{Diviser par un nombre revient à multiplier par son inverse.}

\exemple{
\begin{center}
\begin{tikzpicture}
    \node at (0,0) {$\dfrac{2}{3} \div \dfrac{4}{5} = \dfrac{2}{3} \times \dfrac{5}{4} = \dfrac{10}{12}$};
    \draw[rouge] (0.1,-0.6) rectangle (1.1,0.6);
    \draw[rouge] (-1.85,-0.6) rectangle (-0.85,0.6);
    \draw[rouge,thick,-latex] (-1.35,-0.65) arc (-180:0:1 and 0.5);
    \node at (0,-1.5) {la fraction est inversée};
    \node at (0,-2) {et le $\div$ devient $\times$};
\end{tikzpicture}
\end{center}
}


\newcommand{\frandom}{\directlua{
    local a = math.random(1,20)
    local b = math.random(1,20)
    local c = math.random(1,20)
    local d = math.random(1,20)
    tex.print("\\dfrac{" .. a .. "}{" .. b .. "} + \\dfrac{" .. c .. "}{" .. d .. "} = ")
}}

\newcommand{\trandom}{\directlua{
    local a = math.random(1,20)
    local b = math.random(1,20)
    local c = a * math.random(1,9)
    tex.print("\\dfrac{" .. a .. "}{" .. b .. "} = \\dfrac{" .. c .. "}{" .. "...}")
}}

\renewcommand{\labelitemi}{$\circ$}

\clearpage
\setlength{\itemsep}{3em}
\begin{multicols}{2}
\begin{itemize}
    \item $\frandom$
    \item $\frandom$
    \item $\frandom$
    \item $\frandom$
    \item $\frandom$
    \item $\frandom$
    \item $\frandom$
    \item $\frandom$
    \item $\frandom$
    \item $\frandom$
    \item $\frandom$
    \item $\frandom$
    \item $\frandom$
    \item $\frandom$
    \item $\frandom$
    \item $\frandom$
\end{itemize}
\end{multicols}

\vspace{0.5cm}

\begin{multicols}{3}
\begin{itemize}
    \item $\trandom$
    \item $\trandom$
    \item $\trandom$
    \item $\trandom$
    \item $\trandom$
    \item $\trandom$
    \item $\trandom$
    \item $\trandom$
    \item $\trandom$
    \item $\trandom$
    \item $\trandom$
    \item $\trandom$
    \item $\trandom$
    \item $\trandom$
    \item $\trandom$
    \item $\trandom$
    \item $\trandom$
    \item $\trandom$
    \item $\trandom$
    \item $\trandom$
    \item $\trandom$
\end{itemize}
\end{multicols}


\end{document}