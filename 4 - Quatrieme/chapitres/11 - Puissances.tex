\documentclass[../Cours.tex]{subfiles}
\usepackage{multicol}

\begin{document}

\begin{luacode}
    function nom_prefixe(exposant_prefixe)
        if exposant_prefixe == 15 then return "peta" end
        if exposant_prefixe == 12 then return "tera" end
        if exposant_prefixe == 9 then return "giga" end
        if exposant_prefixe == 6 then return "mega" end
        if exposant_prefixe == 3 then return "kilo" end
        if exposant_prefixe == 2 then return "hecto" end
        if exposant_prefixe == 1 then return "deca" end
        if exposant_prefixe == -1 then return "deci" end
        if exposant_prefixe == -2 then return "centi" end
        if exposant_prefixe == -3 then return "milli" end
        if exposant_prefixe == -6 then return "micro" end
        if exposant_prefixe == -9 then return "nano" end
        if exposant_prefixe == -12 then return "pico" end
        if exposant_prefixe == -15 then return "femto" end
    end

    function prefixe(exposant_prefixe)
        if exposant_prefixe == 0 then return "" end
        return "\\" .. nom_prefixe(exposant_prefixe)
    end

    function parenthese_nombre_negatif(n)
        if n < 0 then return "(-" .. math.abs(n) .. ")" 
        else return math.abs(n) end
    end

    function combinaison_puissance(a, b)
        local aa = parenthese_nombre_negatif(a)
        local bb = parenthese_nombre_negatif(b)
        local cc = parenthese_nombre_negatif(a+b)
        return "(on combine les puissances de 10 $\\Rightarrow " .. aa .. " + " .. bb .. " = " .. cc .. "$)"
    end  
    
    function conversion_puissances(nombre, unite, exposant_prefixe1, exposant_prefixe2)
        tex.print("Convertir \\qty{".. nombre .. "}{" .. prefixe(exposant_prefixe1) .. unite .. "} en \\unit{" .. prefixe(exposant_prefixe2) .. unite .. "}\\\\")
        tex.print("\\renewcommand{\\arraystretch}{1.65}")
        tex.print("\\begin{tabularx}{\\linewidth}{{>{$}l<{$}}R}")

        local i = string.find(nombre, "e")
        local num = tonumber(string.sub(nombre, 1, i-1))
        local expo = tonumber(string.sub(nombre, i+1, -1))

        -- étape 0 : énoncé
        local afficher_expo = "\\times 10^{" .. expo .. "}"
        if expo == 0 then afficher_expo = "" end
        tex.print("\\textcolor{rouge}{\\num{" .. num .. "}} " .. afficher_expo .. "~\\unit{" .. prefixe(exposant_prefixe1) .. unite .. "}&\\\\")

        local decal = math.floor(math.log(num, 10))
        local direction = "droite"
        if decal < 0 then direction = "gauche" end 
        local mantisse = num / math.pow(10,decal)

        if decal ~= 0 then
            if expo ~= 0 then
                -- étape 1 : passage à l'écriture scientifique
                tex.print("=\\textcolor{rouge}{\\num{" .. mantisse .. "} \\times 10^{" .. decal .. "}} \\times 10^{" .. expo .. "}~\\unit{" .. prefixe(exposant_prefixe1) .. unite  .."} & \\makecell[l]{\\textcolor{rouge}{(passage à l'écriture scientifique :} \\\\ \\textcolor{rouge}{la virgule est décalée " .. math.abs(decal) .. " fois vers la " .. direction .. " $\\Rightarrow \\times 10^{" .. decal .. "}$)}}\\\\")
        
                -- étape 2 : combinaison des puissances
                tex.print("=\\num{" .. mantisse .. "} \\times 10^{" .. expo + decal .. "}~\\unit{\\textcolor{vert}{" .. prefixe(exposant_prefixe1) .. "}" .. unite .. "}&" .. combinaison_puissance(decal, expo) .. "\\\\")
                expo = expo + decal
            else
                -- étape 1 : passage à l'écriture scientifique
                tex.print("=\\textcolor{rouge}{\\num{" .. mantisse .. "} \\times 10^{" .. decal .. "}}~\\unit{\\textcolor{vert}{" .. prefixe(exposant_prefixe1) .. "}" .. unite  .."} & \\makecell[l]{\\textcolor{rouge}{(passage à l'écriture scientifique :} \\\\ \\textcolor{rouge}{la virgule est décalée " .. math.abs(decal) .. " fois vers la " .. direction .. " $\\Rightarrow \\times 10^{" .. decal .. "}$)}}\\\\")
                expo = decal
            end
        end

        afficher_expo = "\\times 10^{" .. expo .. "}"
        if expo == 0 then afficher_expo = "" end

        if exposant_prefixe1 ~= 0 then
            -- étape 3 : remplacement du préfixe
            tex.print("=\\num{" .. mantisse .. "}" .. afficher_expo .. " \\times \\textcolor{vert}{10^{" .. exposant_prefixe1 .. "}}~\\unit{" .. unite .. "}&(dans la tableau, \\textcolor{vert}{" .. nom_prefixe(exposant_prefixe1) .. " vaut $10^{" .. exposant_prefixe1 .. "}$})\\\\")
    
            if expo ~= 0 then
                -- étape 4 : combinaison des puissances
                tex.print("=\\num{" .. mantisse .. "} \\times 10^{" .. expo + exposant_prefixe1 .. "}~\\unit{" .. unite .. "}&" .. combinaison_puissance(expo, exposant_prefixe1) .. "\\\\")
                expo = expo + exposant_prefixe1
            end
            expo = exposant_prefixe1
        end

        afficher_expo = "\\times 10^{" .. expo .. "}"
        if expo == 0 then afficher_expo = "" end

        if exposant_prefixe2 ~= 0 then 
            -- étape 5 : ajout du nouveau préfixe
            tex.print("=\\num{" .. mantisse .. "}" .. afficher_expo .. " \\times \\textcolor{bleu}{10^{" .. -exposant_prefixe2 .. "}}~\\unit{\\textcolor{bleu}{" .. prefixe(exposant_prefixe2) .. "}" .. unite .. "}&(dans la tableau, \\qty{1}{" .. prefixe(exposant_prefixe2) .. unite .. "} vaut $10^{" .. exposant_prefixe2 .. "}$~\\unit{" .. unite .. "}, \\textcolor{bleu}{donc \\qty{1}{" .. unite .."} vaut $10^{" .. -exposant_prefixe2 .. "}$~\\unit{" .. prefixe(exposant_prefixe2) .. unite .. "}})\\\\")

            if expo ~= 0 then
                -- étape 6 : combinaison des puissances
                tex.print("=\\num{" .. mantisse .. "} \\times 10^{" .. expo - exposant_prefixe2 .. "}~\\unit{" .. prefixe(exposant_prefixe2) .. unite .. "}&" .. combinaison_puissance(expo, -exposant_prefixe2) .. "\\\\")
                expo = expo - exposant_prefixe2
            end
        end
        
        tex.print("\\end{tabularx}")
        tex.print("\\renewcommand{\\arraystretch}{1}")
    end
\end{luacode}

%%%%%%%%%%%%%%%%%%%%%%%%%%%%%%%%%%%%%%%%%%%%%%%%%%%%%%%%%%%%%%%%%%%%%

\chapitre{Puissances}

\partie{Les puissances de 10}
\souspartie{De la multiplication aux puissances}

\definition{Soit $n$ un nombre entier positif non nul. \\$10^n$ (se lit << 10 puissance $n$ >>) est le produit de $n$ facteurs égaux à 10.}

\exemple{%
\begin{itemize}%
    \item $10^9 = \textcolor{vert}{10 \times 10 \times 10 \times 10 \times 10 \times 10 \times 10 \times 10 \times 10} = \num{1000000000}$
    \item $10^4 = \textcolor{vert}{10 \times 10 \times 10 \times 10} = \num{10000}$
    \item $10^5 = \textcolor{vert}{10 \times 10 \times 10 \times 10 \times 10} = \num{100000}$
    \item $10^2 = \textcolor{vert}{10 \times 10} = \num{100}$
    \item $10^0 = 1$
\end{itemize}%
}

\formule{$$10^{-n} = \dfrac{1}{10^n} ~~~~~~ 10^n \times 10^p = 10^{n+p} ~~~~~~ \dfrac{10^n}{10^p} = 10^{n-p} ~~~~~~ \left(10^n\right)^p = 10^{n\times p} $$}

\begin{listedexemples}
\begin{multicols}{2}
    \item $10^{-2} = \dfrac{1}{10^2} = \dfrac{1}{100} = 0,01$
    \item $10^3\times 10^2 = 10^{3+2} = 10^5$
    \item $10^2 \times 10^4 = 10^{2+4} = 10^6$
    \item $\dfrac{10^4}{10^1} = 10^{4-1} = 10^3$
    \item $10^{-5} = \dfrac{1}{10^5} = \dfrac{1}{\num{100000}} = \num{0,00001}$
    \item $10^3 \times 10^{-2} = 10^{3 + (-2)} = 10^1$
    \item $\left(10^2\right)^3 = 10^{2 \times 3} = 10^6$
    \item $\left(10^4\right)^{-2} = 10^{4 \times (-2)} = 10^{-8}$ 
\end{multicols}
\end{listedexemples}


\clearpage
\souspartie{Écriture scientifique d'un nombre décimal}

\methode{Soit $n$ un nombre entier positif.
\begin{itemize}
    \item Multiplier un nombre par $10^n$, c'est décaler la virgule de $n$ chiffres vers la droite.
    \item Multiplier un nombre par $10^{-n}$, c'est décaler la virgule de $n$ chiffres vers la gauche.
\end{itemize}}

\begin{listedexemples}
\begin{multicols}{2}
\begin{luacode}
    function decalage_virgule(nombre, decal)
        local res = nombre * math.pow(10, decal)
        if res == math.floor(res) then res = math.floor(res) end
        tex.print("$\\num{" .. nombre .. "} \\times 10^{" .. decal .. "} = \\num{" .. res .. "}$")
    end
    item() decalage_virgule(1.234, 2)
    item() decalage_virgule(96.76, -3)
    item() decalage_virgule(39.42649, 4)
    item() decalage_virgule(0.003825, -1)
    item() decalage_virgule(1.9274, 5)
    item() decalage_virgule(10.218, -3)
\end{luacode}
\end{multicols}
\end{listedexemples}

\remarque{Un nombre décimal peut s'écrire de plusieurs façons à l'aide de puissances de 10.}

\definition{L'écriture scientifique d'un nombre décimal non nul ($\neq 0$) est \emph{l'unique} écriture de la forme $a \times 10^n$ où :
\begin{itemize}
    \item $a$ est un nombre décimal n'ayant qu'un chiffre autre que 0 avant la virgule, c'est-à-dire $1 \infeg a < 10$
    \item $n$ est un nombre entier relatif
\end{itemize}}

\begin{listedexemples}
\begin{multicols}{2}
\begin{luacode}
    function ecriture_scientifique(nombre)
        local exposant = math.floor(math.log(nombre, 10))
        local res = nombre / math.pow(10, exposant)
        tex.print("$\\num{" .. nombre .. "} = \\num{" .. res .. "} \\times 10^{" .. exposant .. "}$")
    end
    item() ecriture_scientifique(120546)
    item() ecriture_scientifique(0.436)
    item() ecriture_scientifique(540000)
    item() ecriture_scientifique(0.00876)
    item() ecriture_scientifique(0.0715)
    item() ecriture_scientifique(12.34)
    item() ecriture_scientifique(0.00016)
    item() ecriture_scientifique(6.725)
\end{luacode}
\end{multicols}
\end{listedexemples}

\clearpage
\souspartie{Système International (SI)}

\underline{Tableau des préfixes du SI :}

\color{noir}
{
\setlength{\tabcolsep}{0pt}
\tiny
\newcommand*\rot{\rotatebox{90}}
\begin{center}
\begin{tabularx}{\textwidth}{|*{21}{C|}}\hline
    \multicolumn{3}{|c}{Milliards} & \multicolumn{3}{|c}{Million} & \multicolumn{3}{|c}{Milliers} & \multicolumn{3}{|c}{Unité} & \multicolumn{3}{|c}{Millièmes} & \multicolumn{3}{|c}{Millionièmes} & \multicolumn{3}{|c|}{Milliardièmes}  \\\hline
    & & giga & & & méga & & & kilo & hecto & déca & & déci & centi & milli & & & micro & & & nano \\\hline
    & & $G$ & & & $M$ & & & $k$ & $h$ & $da$ & & $d$ & $c$ & $m$ & & & $\mu$ & & & $n$\\\hline
    & & $10^9$ & \hspace{-3em}\phantom{$10^{1^2}$}& & $10^6$ & & & $10^3$ & $10^2$ & $10^1$ & $10^0$ & $10^{-1}$ & $10^{-2}$ & $10^{-3}$ & & & $10^{-6}$ & & & $10^{-9}$ \\\hline
    \rot{centaines de milliards} & \rot{dizaines de milliards} & \rot{milliards} & \rot{centaines de millions} & \rot{dizaines de millions} & \rot{millions} & \rot{centaines de milliers} & \rot{dizaines de milliers} & \rot{milliers} & \rot{centaines} & \rot{dizaines} & \rot{unités} & \rot{dixièmes} & \rot{centièmes} & \rot{millièmes} & \rot{dix-millièmes} & \rot{cent-millièmes} & \rot{millionièmes} & \rot{dix-millionièmes} & \rot{cent-millionièmes} & \rot{milliardièmes} \\\hline
\end{tabularx}
\end{center}
}

\underline{Tableau des sept unités de base du SI :}

\begin{center}
\begin{tabularx}{\textwidth}{|X|c|c|}\hline
    Grandeur physique & Unité & Symbole de l'unité\\\hline
    masse & kilogramme & kg\\
    longueur & mètre & m\\
    temps & seconde & s\\
    température thermodynamique & kelvin & K\\
    intensité du courant électrique & ampère & A\\
    quantité de matière & mole & mol\\
    intensité lumineuse & candela & cd\\\hline
\end{tabularx}
\end{center}

\color{bleu}

\begin{listedexemples}
    \small
    \item[] \fbox{Cas n°1}
    \begin{luacode}
        item() conversion_puissances("1e0", "\\octet", 9, 0)
        item() conversion_puissances("30e0", "\\metre", 3, 0)
    \end{luacode}
    \clearpage
    \item[] \fbox{Cas n°2}
    \begin{luacode}
        item() conversion_puissances("5e0", "\\volt", 0, -3)
        item() conversion_puissances("101300e0", "\\pascal", 0, 2)
    \end{luacode}
    \item[] \fbox{Cas n°3}
    \begin{luacode}
        item() conversion_puissances("398.76e-6", "\\volt", -9, 9)
        item() conversion_puissances("0.00078e-1", "\\metre", -1, 1)
    \end{luacode}
\end{listedexemples}

\clearpage
\partie{Puissances à base quelconque}

\remarque{$a$ est un nombre entier non nul
\begin{itemize}
    \item Par convention, $a^0=1$
    \item $a^2$ se lit << $a$ au carré >>
    \item $a^3$ se lit << $a$ au cube >>
\end{itemize}}


%%%%%%%%%%%% EXERCICES %%%%%%%%%%%%%%%%
\clearpage
\EXERCICES
\begin{questions}
    \exercicetitre{Écrire les nombres suivants avec la notation scientifique}\vspace{-1.5em}
        \begin{multicols}{2}
            \question $\num{34,5} = $
            \question $\num{17,897654} =$
            \question $\num{0,0000123} =$
            \question $\num{9823,12} =$
            \question $\num{0,100234010} =$
            \question $\num{2871,9273} =$
        \end{multicols}

    \exercicetitre{Écrire les nombres suivants avec l'écriture décimale}\vspace{-1.5em}
        \begin{multicols}{2}
            \question $4,2 \times 10^3 = $
            \question $6,07 \times 10^{-2} =$
            \question $4,23 \times 10^2 =$
            \question $8,923 \times 10^{-3} =$
            \question $9,98 \times 10^{2} = $
            \question $2 \times 10^{6} = $
        \end{multicols}

    \exercicetitre{Convertir les quantités suivantes en mètres}\vspace{-1.5em}
        \begin{multicols}{2}
            \question $\qty{4,2}{\kilo\metre} =$
            \question $\qty{2,5}{\giga\metre} =$
            \question $\qty{9,8}{\milli\metre} =$
            \question $\qty{8,1}{\nano\metre} =$
        \end{multicols}

    \exercicetitre{Convertir les quantités suivantes selon l'unité demandée}\vspace{-1.5em}
        \begin{multicols}{2}
            \question $\qty{2,3}{\metre} \mbox{~en~} \unit{\centi\metre}$
            \question $\qty{0,02}{\metre} \mbox{~en~} \unit{\kilo\metre}$
            \question $\qty{0,002}{\metre} \mbox{~en~} \unit{\nano\metre}$
            \question $\qty{18,258}{\metre} \mbox{~en~} \unit{\mega\metre}$
            \question $\qty{0,035}{\metre} \mbox{~en~} \unit{\deca\metre}$
            \question $\qty{9275,729}{\metre} \mbox{~en~} \unit{\hecto\metre}$
        \end{multicols}

    \exercicetitre{Convertir les quantités suivantes}\vspace{-1.5em}
        \begin{multicols}{2}
            \question $\qty{398.76e-6}{\nano\volt}$ en \unit{\giga\volt}
            \question $\qty{0.00078e-1}{\deci\metre}$ en \unit{\deca\metre}
            \question $\qty{1.35e-9}{\mega\gram}$ en \unit{\centi\gram}
            \question $\qty{0.00056e-5}{\kilo\metre}$ en \unit{\nano\metre}
            \question $\qty{1.0058e-9}{\micro\watt}$ en \unit{\hecto\watt}
            \question $\qty{9.61e18}{\milli\metre}$ en \unit{\mega\metre}
            \question $\qty{49005.4e7}{\micro\litre}$ en \unit{\centi\litre}
            \question $\qty{8.56e9}{\nano\metre}$ en \unit{\giga\metre}
        \end{multicols}

    \exercicetitre{Calculs entre nombres en écriture scientifique}\vspace{-1.5em}
        \begin{multicols}{2}
            \question $\num{2e5} \times \num{7.5e6} = $
            \question $\dfrac{\num{3.5e-6}}{\num{14e-2}} = $
            \question $\num{13e21} + \num{2e19} = $
            \question $\num{4.8e-10} - \num{2e-11} = $
        \end{multicols}

    \exercice\\
        $ABCD$ est un carré tel que $AB = \qty{4e25}{\milli\metre}$.\\
        $EFGH$ est un rectangle tel que $EF = \qty{8,9e24}{\milli\metre}$ et $EG = \qty{2,5e25}{\milli\metre}$.
        \question Entre $ABCD$ et $EFGH$, lequel a la plus grande aire ?
        \question Entre $ABCD$ et $EFGH$, lequel a le plus grand périmètre ?

    \exercicetitre{Molécules d'eau}
        Une molécule d'eau est constituée d'un atome d'oxygène et de deux atomes d'hydrogène.\\
        Un atome d'oxygène pèse \qty{2.7e-26}{kg}.\\
        Un atome d'hydrogène pèse \qty{0.17e-26}{kg}.\\
        \question Calculer le nombre de molécules d'eau dans une bouteille de \qty{1.5}{\litre}.
        \question Convertir cette valeur en mole, sachant que $\qty{1}{\mol} = \qty{6.022e23}{éléments}$. 
\end{questions}

%%%%%%%%%%%%%% CORRECTIONS %%%%%%%%%%%%%%%%%
\clearpage
\CORRECTIONS
\begin{questions}
    \exercicetitre{Écrire les nombres suivants avec la notation scientifique}\vspace{-1.5em}
    \begin{multicols}{2}
    \begin{luacode}
        question() ecriture_scientifique(34.5)
        question() ecriture_scientifique(17.897654)
        question() ecriture_scientifique(0.0000123)
        question() ecriture_scientifique(9823.12)
        question() ecriture_scientifique(0.100234010)
        question() ecriture_scientifique(2871.9273)
    \end{luacode}
    \end{multicols}
    
    \exercicetitre{Écrire les nombres suivants avec l'écriture décimale}\vspace{-1.5em}
    \begin{multicols}{2}
    \begin{luacode}
        question() decalage_virgule(4.2, 3)
        question() decalage_virgule(6.07, -2)
        question() decalage_virgule(4.23, 2)
        question() decalage_virgule(8.923, -3)
        question() decalage_virgule(9.98, 2)
        question() decalage_virgule(2, 6)
    \end{luacode}
    \end{multicols}
    
    \exercicetitre{Convertir les quantités suivantes en mètres}
    \begin{luacode}
        question() conversion_puissances("4.2e0", "\\metre", 3, 0)
        question() conversion_puissances("2.5e0", "\\metre", 9, 0)
        question() conversion_puissances("9.8e0", "\\metre", -3, 0)
        question() conversion_puissances("8.1e0", "\\metre", -9, 0)
    \end{luacode}
    
    \exercicetitre{Convertir les quantités suivantes selon l'unité demandée}
    \begin{luacode}
        question() conversion_puissances("2.3e0", "\\metre", 0, -2)
        question() conversion_puissances("0.02e0", "\\metre", 0, 3)
        question() conversion_puissances("0.002e0", "\\metre", 0, -9)
        question() conversion_puissances("18.258e0", "\\metre", 0, 6)
        question() conversion_puissances("0.035e0", "\\metre", 0, 1)
        question() conversion_puissances("9275.729e0", "\\metre", 0, 2)
    \end{luacode}
    
    \exercicetitre{Convertir les quantités suivantes}
    \begin{luacode}
        question() conversion_puissances("398.76e-6", "\\volt", -9, 9)
        question() conversion_puissances("0.00078e-1", "\\metre", -1, 1)
        question() conversion_puissances("1.35e-9", "\\gram", 6, -2)
        question() conversion_puissances("0.00056e-5", "\\metre", 3, -9)
        question() conversion_puissances("1.0058e-9", "\\watt", -6, 2)
        question() conversion_puissances("9.61e18", "\\metre", -3, 6)
        question() conversion_puissances("49005.4e7", "\\litre", -6, -2)
        question() conversion_puissances("8.56e9", "\\metre", -9, 9) 
    \end{luacode}

    \exercicetitre{Calculs entre nombres en écriture scientifique}
        \noindent\begin{minipage}{0.5\linewidth}
        \begin{align*}
            \mbox{\textbf{Q1)~}} \num{2e5} \times \num{7.5e6} &= \num{2} \times \num{7.5} \times 10^5 \times 10^6 \\
            &= 15 \times 10^{11} \\
            &= \num{1.5e1} \times 10^{11} \\
            &= \num{1.5e12}
        \end{align*}
        \end{minipage}
        \begin{minipage}{0.5\linewidth}
        \begin{align*}
            \mbox{\textbf{Q2)~}} \dfrac{\num{3.5e-6}}{\num{14e-2}} &= \dfrac{\num{3.5}}{14} \times \dfrac{10^{-6}}{10^{-2}} \\
            &= \num{0.25} \times 10^{(-6)-(-2)} \\
            &= \num{0.25} \times 10^{-4} \\ 
            &= \num{2.5e-1} \times 10^{-4} \\
            &= \num{2.5e-5}
        \end{align*}
        \end{minipage}

        \noindent\begin{minipage}{0.5\linewidth}
        \begin{align*}
            \mbox{\textbf{Q3)~}} &\num{13e21} + \num{2e19} \\
            &= \num{13e2} \times \textcolor{rouge}{10^{19}} + 2 \times \textcolor{rouge}{10^{19}} \\
            &= \textcolor{rouge}{10^{19}} \left( 13 \times 10^2 + 2 \right) \\
            &= 10^{19} \left( 1300 + 2 \right) \\
            &= 1302 \times 10^{19} \\
            &= \num{1.302e3} \times 10^{19} \\
            &= \num{1.302e22}
        \end{align*}
        \end{minipage}
        \begin{minipage}{0.5\linewidth}
        \begin{align*}
            \mbox{\textbf{Q4)~}} &\num{4.8e-10} - \num{2e-11} \\
            &= \num{4.8} \times \textcolor{rouge}{10^{-10}} - \num{2e-1} \times \textcolor{rouge}{10^{-10}} \\
            &= 10^{-10} \left( \num{4.8} - 2 \times 10^{-1} \right) \\
            &= 10^{-10} \left( \num{4.8} - \num{0.2} \right) \\
            &= \num{4.6e-10}
        \end{align*}
        \end{minipage}

    \exercice\\
        \question Calculons les aires de $ABCD$ et $EFGH$\\
        \noindent\begin{minipage}{0.5\linewidth}
        \begin{align*}
            \mathcal{A}_{ABCD} &= \mbox{côté} \times \mbox{côté} \\
            &= AB \times AB \\
            &= 4 \times 10^{25}~\unit{mm} \times 4 \times 10^{25}~\unit{mm} \\
            &= 4 \times 4 \times 10^{25} \times 10^{25}~\unit{\milli\metre\squared} \\
            &= 16 \times 10^{50}~\unit{\milli\metre\squared} \\
            &= 1,6 \times 10^1 \times 10^{50}~\unit{\milli\metre\squared} \\
            &= 1,6 \times 10^{51}~\unit{\milli\metre\squared}
        \end{align*}\end{minipage}
        \begin{minipage}{0.5\linewidth}
        \begin{align*}
            \mathcal{A}_{EFGH} &= \mbox{longueur} \times \mbox{largeur} \\
            &= EF \times EG \\
            &= \qty{8.9e24}{mm} \times \qty{2.5e25}{mm} \\
            &= \num{8.9} \times \num{2.5} \times 10^{24} \times 10^{25} ~\unit{mm\squared} \\
            &= \qty{22.25e49}{mm\squared} \\
            &= \qty{2.225e1} \times 10^{49} ~\unit{mm\squared} \\
            &= \qty{2.225e50}{mm\squared}
        \end{align*}
        \end{minipage}

        \centerline{Donc $\mathcal{A}_{ABCD} > \mathcal{A}_{EFGH}$.}

        \question Calculons les périmètres de $ABCD$ et $EFGH$\\
        \noindent\begin{minipage}{0.5\linewidth}
        \begin{align*}
            \mathcal{P}_{ABCD} &= 4 \times \mbox{côté} \\
            &= 4 \times AB \\
            &= 4 \times \qty{4e25}{mm} \\
            &= 16 \times 10^{25}~\unit{mm} \\
            &= \num{1.6e1} \times 10^{25}~\unit{mm} \\
            &= \qty{1.6e26}{mm}
        \end{align*}
        \end{minipage}
        \begin{minipage}{0.5\linewidth}
        \begin{align*}
            \mathcal{P}_{EFGH} &= 2 \left( \mbox{longueur} + \mbox{largeur} \right) \\
            &= 2 \left( EF + EG \right) \\
            &= 2 \left( \qty{8.9e24}{mm} + \qty{2.5e25}{mm} \right) \\
            &= 2 \left( \qty{8.9e24}{mm} + \num{2.5e24} \times 10^1 ~\unit{mm} \right) \\
            &= 2 \times 10^{24} \left( \num{8.9} + \num{2.5e1} \right)~\unit{mm} \\
            &= 2 \times 10^{24} \left( \num{8.9} + 25 \right)~\unit{mm} \\
            &= 2 \times 10^{24} \times \qty{33.9}{mm} \\
            &= \qty{67.8e24}{mm} \\
            &= \num{6.78e1} \times 10^{24} ~\unit{mm} \\
            &= \qty{6.78e25}{mm}
        \end{align*}
        \end{minipage}

        \centerline{Donc $\mathcal{P}_{ABCD} > \mathcal{P}_{EFGH}$.}

    \exercicetitre{Molécules d'eau}
        \question Tout d'abord, il faudrait calculer la masse d'une molécule d'eau.
        \begin{align*}
            \mbox{masse} (H_2O) &= 2 \times \mbox{masse (hydrogène)} + \mbox{masse (oxygène)} \\
            m(H_2O) &= 2 \times m(H) + m(O) \\
            &= 2 \times \qty{0.17e-26}{\kilo\gram} + \qty{2.7e-26}{\kilo\gram} \\
            &= \qty{0.34e-26}{\kilo\gram} + \qty{2.7e-26}{\kilo\gram} \\
            &= 10^{-26} \left( \num{0.34} + \num{2.7} \right) ~\unit{\kilo\gram}\\
            &= \qty{3.04e-26}{\kilo\gram}
        \end{align*}

        Il faut ensuite savoir que \qty{1}{\litre~d'eau} correspond à \qty{1}{\kilo\gram~d'eau}. Donc, sachant qu'une molécule d'eau pèse \qty{3.04e-26}{\kilo\gram}, on se demande combien y a-t-il de molécules d'eau dans \qty{1.5}{\kilo\gram}. Pour cela, on effectue le calcul suivant :

        \begin{align*}
            \mbox{nombre de molécules} &= \frac{\mbox{masse totale}}{\mbox{masse d'une molécule}}\\[1em]
            n(H_2O) &= \dfrac{m_{\mbox{totale}}}{m(H_2O)}\\
            &= \dfrac{\qty{1.5}{\kilo\gram}}{\qty{3.04e-26}{\kilo\gram}}\\
            &= \dfrac{1,5}{3,04} \times \dfrac{1}{10^{-26}} ~ \dfrac{\unit{kg}}{\unit{kg}} \\
            &= \num{0.493e26} \\
            &= \num{4.93e-1} \times 10^{26} \\
            &= \num{4.93} \times 10^{25}
        \end{align*}

        En conclusion, dans \qty{1.5}{\litre~d'eau}, il y a \qty{4.93e25}{molécules~d'eau}.

        \question 
        \begin{center}
            \begin{tabularx}{0.35\linewidth}{|c|C|}\hline
                1 mol & \qty{6.022e23}{molécules} \\\hline
                ? & \qty{4.93e25}{molécules}\\ \hline
            \end{tabularx}
        \end{center}
        \begin{align*}
            n &= \dfrac{\qty{4.93e25}{molécules}}{\qty{6.022e23}{molécules}} \\
            &= \dfrac{\num{4.93}}{\num{6.022}} \times \dfrac{10^{25}}{10^{23}} \\
            &= \num{0.819} \times 10^2 \\
            &= \num{8.19e-1} \times 10^2 \\
            &= \num{8.19e1} \\
            &= \qty{81.9}{mol}
        \end{align*}

        En conclusion, dans \qty{1.5}{\litre~d'eau}, il y a \qty{81.9}{mol}.
\end{questions}

\end{document}