\documentclass[10pt,addpoints]{exam}

\usepackage[margin=1cm]{geometry}
\usepackage{amsmath}
\usepackage{multicol}
\usepackage{siunitx}
\usepackage{soul}
\usepackage{fancybox}

\begin{document}

\begin{center}
    \Huge{Contrôle sur les puissances}
\end{center}

\newcommand{\sol}[1]
{%
\ifprintanswers \textbf{#1} \fi%
}

\unframedsolutions

\begin{questions}
\qformat{\textbf{Exercice~\thequestion)}\hfill(\thepoints)}

\question[2] Compléter le tableau ci-dessous.

\begin{table}[h!]
    \centering
    \begin{tabular}{|l|c|c|c|c|c|c|c|c|c|c|}\hline
    Préfixe & giga & méga & kilo & hecto & déca & déci & centi & milli & \sol{micro} & nano  \\\hline
    Symbole & G & \sol{M} & k & h & \sol{da} & d & c & m & $\mu$ & n \\\hline
    Puissance de 10\phantom{$1^{2^3}$} & $10^9$ & $10^6$ & $10^3$ & $10^2$ & $10^1$ & $10^{-1}$ & \sol{$10^{-2}$} & $10^{-3}$ & $10^{-6}$ & $10^{-9}$ \\\hline
    \end{tabular}
\end{table}

\question[4] \textsc{Calculs avec des puissances}
\begin{parts}
    \part Donner chaque résultat sous la forme d'une puissance de 10
    \begin{subparts}
        \subpart $10^2 \times 10^3 = $ \fillin[$10^5$]
        \subpart $10^6 \times 10^9 \times 10^{11} = $ \fillin[$10^{26}$]
        \subpart $\left(10^4\right)^9 = $ \fillin[$10^{36}$]
        \subpart $\dfrac{10^4}{10^2} = $  \fillin[$10^2$]
    \end{subparts}
    \part Donner chaque résultat sous la forme d'une puissance de 2
    \begin{subparts}
        \subpart $2^3 \times 2^4 = $ \fillin[$2^7$]
        \subpart $\left(2^5\right)^8 = $ \fillin[$2^{40}$]
        \subpart $\left(2^3\right)^4 \times 2^6 = $ \fillin[$2^{18}$]
        \subpart $\dfrac{2^2 \times 2^4}{2^7} = $ \fillin[$2^{-1}$]
    \end{subparts}
\end{parts}

\newcommand{\vrai}[1]{\ifprintanswers \ovalbox{#1} \else #1 \fi}
\newcommand{\faux}[1]{\ifprintanswers \st{#1} \else #1 \fi}

\question[9] \textsc{Écriture scientifique}
\begin{parts}
    \part Compléter la phrase suivante : << L'écriture scientifique d'un nombre se compose d'un nombre décimal compris entre \fillin[1] et \fillin[10] multiplié par une puissance de \fillin[10]. >>
    \part Parmi les nombres suivants, entourer ceux qui sont en notation scientifique, et barrer les autres.
    \begin{multicols}{4}
    \begin{subparts}
        \subpart \faux{$12,5 \times 10^4$} 
        \subpart \faux{$0,1 \times 10^{-1}$}
        \subpart \vrai{$1,7 \times 10^2$}
        \subpart \vrai{$9,9 \times 10^0$}
        \subpart \faux{$10 \times 10^1$}
        \subpart \faux{$5,834 \times 8^2$}
        \subpart \faux{$3$}
        \subpart \vrai{$7,1 \times 10^{189}$}
    \end{subparts}
    \end{multicols}
    \part Écrire les nombres suivants sous forme d'écriture scientifique
    \begin{subparts}
        \subpart $123,4 = $ \fillin[$1,234 \times 10^2$]
        \subpart $0,000~045 = $ \fillin[$4,5 \times 10^{-5}$]
        \subpart $17 \times 10^2 = $ \fillin[$1,7 \times 10^3$]
        \subpart $0,8 \times 10^{-2} = $ \fillin[$8 \times 10^{-3}$]
    \end{subparts}
\end{parts}

\question[3] \textsc{Conversion d'unités} (Donner les résultats en écriture scientifique)

\begin{parts}
    \part \qty{32}{km} = \fillin[$3,2 \times 10^{10}$] \unit{\micro\metre}
    \part \qty{3,8}{mm} = \fillin[$3,8  \times 10^{6}$] \unit{\nano\metre}
    \part \qty{4,5}{dam} = \fillin[$4,5 \times 10^{-8}$] \unit{\giga\metre}
\end{parts}


\question[2] \textsc{Utiliser la proportionnalité}
\begin{parts}
    \part Sachant que la vitesse de la lumière est de \qty{300000}{\kilo\metre/s}, quelle est la distance parcourue par la lumière sur une période de 1 semaine. Donner le résultat sous forme d'écriture scientifique. \fillin[$1,8144 \times 10^{11}$~km]
    \part Sachant que 1 mole correspond à $6,02 \times 10^{23}$ atomes, à combien d'atomes correspond $1,5 \times 10^{-2}$ mole ? Donner le résultat sous forme d'écriture scientifique. \fillin[]
\end{parts}

\begin{center}
    \gradetable[h][questions]
\end{center}

\end{questions}
\end{document}