\documentclass[12pt]{article}
\usepackage[margin=1.5cm]{geometry}
\usepackage{xcolor}
\usepackage{siunitx}
\usepackage{frenchmath}
\usepackage{amsmath}
\usepackage{tabularx}
\begin{document}

\thispagestyle{empty}

\textbf{Exercice 1 : }
{\color{red} B - B - D - B } (3 points par bonne réponse)

\textbf{Exercice 2 : }

Q1) {\color{red} $SH = 2064-589 = \qty{1475}{\metre}$} (4 points)

Q2) {\color{red} 
    \begin{itemize}
        \item[(2pts)] Le triangle $DHS$ est rectangle en H. (l'hypoténuse est $[DS])$
        \item[(2pts)] D'après le théorème de Pythagore :
        \item[(3pts)] $DH^2 + HS^2 = DS^2$
        \item[] $DH^2 = DS^2 - HS^2$
        \item[(2pts)] $DH^2 = 28.28^2 - 1.475^2$ (conversion \unit{\m} en \unit{\km})
        \item[] $DH^2 = 797.582775$
        \item[(3pts)] $DH = \sqrt{797.582775}$
        \item[] $DH \approx \qty{28.24}{\kilo\metre}$
    \end{itemize}
}

\textbf{Exercice 3 : }

Q1) (8pts = 2tps par calcul) a) Réponses : 41 b) Réponses : 5

Q2) (2pts) {\color{red} En choisissant les nombres 10 et -2, on \emph{obtient le même résultat pour les deux programmes}.}

Q3) a) (6pts = 3pts par calcul) {\color{red} Programme A : \boxed{$3(x-5)-4$} Programme B : \boxed{$3x+11$} }

b) (4pts) {\color{red}$3(x-5)-4 = 3x+11$} (en utilisant la distributivité)

\textbf{Exercice 4 : }

Q1) (2pts) {\color{red} Il faudrait \num{100000} personnes de plus en 2020 pour atteindre 2 millions de visiteurs.}

Q2) (3pts) {\color{red} $\num{1900000} \div 365 \approx 5205$, donc l'affirmation est vraie.}

Q3) {\color{red}
    \begin{itemize}
        \item[(2pt)] Les points A,E,B sont alignés
        \item[(2pt)] Les points A,D,C sont alignés
        \item[(2pt)] Les droites $(DE)$ et $(BC)$ sont parallèles.
        \item[(2pts)] D'après le théorème de Thalès :
        \item[(3pts)] $\frac{AD}{AC} = \frac{AE}{AB} = \frac{DE}{BC}$
        \item[(4pts)] $BC = \frac{1.60 \times 56.25}{2} = \qty{45}{\metre}$ (3pts résultat + 1pt unité)
    \end{itemize}
}

\textbf{Exercice 5 : }

Q1)a) (4pts) {\color{red} 3 employés sont en situation de surpoids ou d'obésité.}

b) (4pts) {\color{red} =B2/(B1*B1)}

Q2)a) (4pts) {\color{red} L'IMC moyen est de 23{,}15.}

b) (4pts) {\color{red} L'IMC médian est de 21.}

c) (4pts) {\color{red} 6 employés ont un IMC supérieur ou égal à 25. Pourcentage : $\frac{6}{41} \approx \qty{14.6}{\%}$.}

\textbf{Exercice 6 : }

Q1) (3pts) {\color{red} \qty{1260}{\gram}}

Q2) {\color{red}
    \begin{itemize}
        \item[(1pt)] rayon = \qty{3}{\cm}
        \item[(1pt)] hauteur = \qty{11}{\cm}
        \item[(2pts)] volume = $99\pi~\unit{\cm\cubed} \approx \qty{311}{\cm\cubed}$
        \item[(2pts)] conversion = \qty{0.311}{\litre}
        \item[(3pts)] $\frac{2.7}{0.311} \approx \num{8.68}$. Donc on pourra remplir 8 pots.
    \end{itemize}
}

\end{document}